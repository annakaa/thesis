\documentclass
[   oneside,         % oneside/twoside : Einseitiger oder zweiseitiger Druck?
    12pt,            % Bezug: 12-Punkt Schriftgre
    DIV15,           % Randaufteilung, siehe Dokumentation "KOMA"-Script
    BCOR17mm,        % Bindekorrektur: Innen 17mm Platz lassen. Copyshop-getestet.
    headsepline,     % Unter Kopfzeile Trennlinie (aus: headnosepline)
    footsepline,     % ber Fuzeile Trennlinie (aus: footnosepline)
    openright,       % Neue Kapitel im zweiseitigen Druck rechts beginnen lassen
    a4paper,         % Seitenformat A4
    abstracton,      % Abstract einbinden
	english,
    listof=totoc,version=first,      % Div. Verzeichnisse ins Inhaltsverzeichnis aufnehmen
    bibliography=totoc,version=first,        % Literaturverzeichnis ins Inhaltsverzeichnis aufnehmen
    titlepage,       % Titelseite aktivieren
    headinclude,     % Seiten-Head in die Satzspiegelberechnung mit einbeziehen
    footexclude,     % Seiten-Foot nicht in die Satzspiegelberechnung mit einbeziehen
    numbers=noenddot,version=first % Gliederungsnummern ohne abschlieenden Punkt darstellen
]   {scrreprt}       % Dokumentenstil: "Report" aus dem KOMA-Skript-Paket

\usepackage[active]{srcltx}
%\usepackage[activate=normal]{pdfcprot} % Optischer Randausgleich -> pdflatex!
\usepackage{ifthen}
%\usepackage[english,ngerman]{babel}
\usepackage[english]{babel}
%\usepackage{ngerman}
%\usepackage[fixlanguage]{babelbib}
%\setbiblanguage{english}
\usepackage[latin1]{inputenc}
\usepackage[T1]{fontenc}
\usepackage[T1]{url}
\usepackage{ae}
\usepackage[final]{graphicx}
\usepackage[automark]{scrpage2}
\usepackage{setspace}
\usepackage{floatflt} 
\usepackage{rotating} 
\usepackage{wrapfig}
\usepackage{subfig}
\usepackage{graphicx}
%\usepackage[first,light]{draftcopy} % Fr Probedruck
\usepackage[plainpages=false,pdfpagelabels,hypertexnames=false]{hyperref}
\usepackage{pdfpages}	%include pdf files
\usepackage{listings} %include sourcecode
\usepackage{color}
\definecolor{commentgreen}{rgb}{0,0.5,0}
\usepackage{multirow}
\usepackage{verbatim}
\usepackage{amsmath}
\usepackage{amsfonts}
\usepackage{enumitem}
\usepackage{setspace}
\usepackage{array}
\usepackage{nomencl}
% \usepackage[table]{xcolor}
\usepackage{longtable}
\usepackage{url}
%\usepackage{paralist}
%\usepackage{subcaption}

%\usepackage{textcomp}  %mit \textcent geht Cent-Symbol




% Tiefe der Kapitelnummerierung beeinflussen
\setcounter{secnumdepth}{3} % Tiefe der Nummerierung
\setcounter{tocdepth}{3}    % Tiefe des Inhaltsverzeichnisses

% Hier in die zweite geschweifte Klammer jeweils
% die persnlichen Daten eintragen:
\newcommand{\artderausarbeitung}{PhD Thesis}
\newcommand{\namedesautors}{Anna Marie Kruspe}
\newcommand{\inventarisierungsnummer}{}

\newcommand{\markup}[1]{\textbf{#1}}

% Seitenlayout festlegen. Hier nichts ndern!
\pagestyle{scrplain}
\ihead[]{\headmark}
\ohead[]{\pagemark}
\chead[]{}
\ifoot[]{\scriptsize \inventarisierungsnummer}
\ofoot[]{\scriptsize \artderausarbeitung\ \namedesautors}
\cfoot[]{}
\renewcommand{\titlepagestyle}{scrheadings}
\renewcommand{\partpagestyle}{scrheadings}
\renewcommand{\chapterpagestyle}{scrheadings}
\renewcommand{\indexpagestyle}{scrheadings}

% Abschnittsweise Nummerierung anstatt fortlaufend. Hier nichts ndern!
\makeatletter
\@addtoreset{equation}{chapter}
\@addtoreset{figure}{chapter}
\@addtoreset{table}{chapter}
\renewcommand\theequation{\thechapter.\@arabic\c@equation}
\renewcommand\thefigure{\thechapter.\@arabic\c@figure}
\renewcommand\thetable{\thechapter.\@arabic\c@table}\makeatother

% Quelltextrahmen, klein. Hier nichts ndern!
\newsavebox{\inhaltkl}
\def\rahmenkl{\sbox{\inhaltkl}\bgroup\small\renewcommand{\baselinestretch}{1}\vbox\bgroup\hsize\textwidth}
\def\endrahmenkl{\par\vskip-\lastskip\egroup\egroup\fboxsep3mm%
\framebox[\textwidth][l]{\usebox{\inhaltkl}}}

% Quelltextrahmen, normale Groesse. Hier nichts ndern!
\newsavebox{\inhalt}
\def\rahmen{\sbox{\inhalt}\bgroup\renewcommand{\baselinestretch}{1}\vbox\bgroup\hsize\textwidth}
\def\endrahmen{\par\vskip-\lastskip\egroup\egroup\fboxsep3mm%
\framebox[\textwidth][l]{\usebox{\inhalt}}}


% Sonstige Befehlsdefinitionen hier ablegen.
\newcommand{\entspricht}{\stackrel{\wedge}{=}}
\definecolor{light-gray}{gray}{0.95}
\addto\captionsenglish{\renewcommand{\contentsname}{Table of Contents}}
%\makenomenclature
\makeglossary

\begin{document}
\onehalfspacing
\clubpenalty=10000
\widowpenalty=10000
\thispagestyle{empty}
%Eigentlich f�r documentclass [11pt] ausgelegt!
\begin{center}
\large{Technische Universit�t Ilmenau}\\
\large{Fakult�t f�r Elektrotechnik und Informationstechnik}
\end{center}

\vspace{0.15cm}

%\begin{figure}[htbp]
%  \centering
%  \subfloat[]{\includegraphics[width=0.4\textwidth]{images/tui_logo.pdf}}
%  \subfloat[]{\includegraphics[width=0.4\textwidth]{images/idmt_logo.pdf}}
%\end{figure}

\begin{figure}[htbp]
\begin{center}$
\begin{array}{cc}
\includegraphics[width=0.4\textwidth]{images/tui_logo} &
\includegraphics[width=0.4\textwidth]{images/idmt_85mm_p334}
\end{array}$
\end{center}
\end{figure}

\vspace{0.6cm}

\begin{center}
%\renewcommand{\baselinestretch}{1.50}\normalsize		%Zeilenabstand
%\vspace{0.25cm}
\textbf{\Large{Application of Automatic Speech Recognition Technologies to Singing}}\\

%\renewcommand{\baselinestretch}{1.00}\normalsize
\end{center}
\vspace{.3cm}

\begin{center}
%Diplomarbeit am Fraunhofer Institut f�r Digitale Medientechnik
Doctoral Thesis
\end{center}






\vspace{3.0cm}
\begin{flushleft}
\small
\renewcommand{\arraystretch}{1.5} 
\begin{tabular}{lll}
\textbf{Submitted by:} & & Anna Marie Kruspe\\
\textbf{Date \& place of birth:} & & July 6, 1987, Halle/Saale\\
%\textbf{Submission date:} & & \\
\textbf{Course of study:} & & Media Technology\\
%\textbf{Specialisation:} & & Audio-Visual Technology\\
\textbf{Matriculation Number:} & & 39909 \\
& & \\
\end{tabular}
\begin{tabular}{lll}
%\textbf{Verantwortlicher Hochschullehrer:} & & Prof. Dr.-Ing. Karlheinz Brandenburg\\
\textbf{Advisor:} & & Prof. Dr.-Ing. Dr. rer. nat. h.c. mult. Karlheinz Brandenburg \\
%\end{tabular}
%\begin{tabular}{lll}
%\textbf{Faculty Advisors:}  & & Advisor 1 \\ & & Advisor 2 \\ & & Advisor 3\\
\end{tabular}
\end{flushleft}





% Inhaltsverzeichnis
\cleardoublepage % Seitenumbruch erzwingen vor nderung des Nummerierungsstils
\footnotesize
%\begin{otherlanguage}[english]
\pagenumbering{roman} % Nummerierung der Seiten ab hier: i, ii, iii, iv...
\pagestyle{scrheadings} % Ab hier mit Kopf- und Fusszeile

\normalsize
%\renewcommand{\abstractname}{Kurzfassung}
%\begin{otherlanguage}{english}
\begin{abstract}
The research field of Music Information Retrieval is concerned with the automatic analysis of musical characteristics. One aspect that has not received much attention so far is the automatic analysis of sung lyrics. On the other hand, the field of Automatic Speech Recognition has produced many methods for the automatic analysis of speech, but those have rarely been employed for singing so far. This thesis analyzes the feasibility of applying various speech recognition methods to singing, and suggests adaptations. In addition, the routes to practical applications for these systems are described. Five tasks are considered: Phoneme recognition, language identification, keyword spotting, lyrics-to-audio alignment, and retrieval of lyrics from sung queries.\\
For the task of phoneme recognition, two large data sets were created. The first one is generated by making speech recordings more ``song-like''. The second one is based on an existing singing data set with matching textual lyrics, which are automatically aligned to the audio and then used for training new acoustic models. The phoneme error rate is improved from $1.08$ to $0.77$ on singing.\\
Two methods for language identification are presented. The first one employs i-vector extraction and achieves an accuracy of $0.73$ on singing. The second method is a completely new approach based on the extraction of phoneme statistics with the new acoustic models. The accuracy is lower at $0.63$, but the approach is promising for systems in which phoneme extraction is performed.\\
For the keyword spotting task, the new acoustic models were integrated into a keyword-filler HMM system. Additionally, duration modeling is applied to the results. In an evaluation with 15 short keywords, an $F_1 $ measure of $0.61$ is obtained.\\
In both lyrics-to-singing alignment and retrieval of textual lyrics from sung queries, an alignment between the audio recording and the text lyrics is performed. For retrieval, the scores for all possible lyrics are compared. A first approach employs ``classic'' Viterbi alignment, and is used for the generation of the previously mentioned singing data set. Two new approaches perform alignment between the lyrics' phonemes and phoneme posteriorgrams extracted with the new acoustic models, either with Dynamic Time Warping or with Levenshtein alignment. The last approach performed best in the \textit{MIREX} 2017 lyrics-to-audio alignment challenge, and achieves accuracies of $0.54$ for short segments and $0.94$ for whole-song queries. Additionally, a practical application to expletive detection in singing is presented.
\end{abstract}
%\end{otherlanguage}
\begin{otherlanguage}{ngerman}
%\renewcommand{\abstractname}{Kurzfassung}
%\newcommand{\dtabstract}{\hyphenpenalty=10000}
%{\dtabstract
\begin{abstract}
Das Gebiet des Music Information Retrieval befasst sich mit der automatischen Analyse von musikalischen Charakteristika. Ein Aspekt, der bisher kaum erforscht wurde, ist dabei der gesungene Text. Auf der anderen Seite werden in der automatischen Spracherkennung viele Methoden f�r die automatische Analyse von Sprache entwickelt, jedoch selten f�r Gesang. Die vorliegende Arbeit untersucht die Anwendung von Methoden aus der Spracherkennung auf Gesang und beschreibt m�gliche Anpassungen. Zudem werden Wege zur praktischen Anwendung dieser Ans�tze aufgezeigt. F�nf Themen werden dabei betrachtet: Phonemerkennung, Sprachenidentifikation, Schlagwortsuche, Text-zu-Gesangs-Alignment und Suche von Texten anhand von gesungenen Anfragen.\\
F�r die Phonemerkennung wurden zwei neue Datens�tze erzeugt. F�r den ersten wurden Sprachaufnahmen k�nstlich �hnlicher zu Gesang gemacht. F�r den zweiten wurden Texte automatisch zu einem vorhandenen Gesangsdatensatz alignt, der dann zum Trainieren neuer akustischer Modelle genutzt wurde. Die Phoneme Error Rate sinkt dabei von $1.08$ auf $0.77$.\\
Zwei Ans�tze f�r die Sprachenidentifikation werden vorgestellt. Im ersten kommt die i-vector-Extraktion zum Einsatz, und die Accuracy liegt bei $0.73$. Die zweite Methode ist ein v�llig neuer Ansatz, bei dem mit Hilfe der neuen akustischen Modelle Phonemstatistiken berechnet werden. Die Accuracy ist hier $0.63$, aber der Ansatz ist vielversprechend f�r Systeme, in denen Phoneme erkannt werden.\\
F�r die Schlagwortsuche wurden die neuen akustischen Modelle in ein Keyword-Filler-HMM-System integriert. Auch Duration Modeling wurde eingesetzt. F�r 15 kurze Schlagworte wird ein $F_1$-Ma� von $0.61$ erzielt.\\
Im Text-zu-Gesangs-Alignment und in der Textsuche wird jeweils ein Alignment zwischen der Audioaufnahme und den Texten berechnet. F�r die Textsuche werden dann die Ergebnisse f�r alle m�glichen Texte verglichen. In einem ersten Versuch wird ``klassisches'' Viterbi-Alignment eingesetzt, u.a. auch zur Erzeugung des erw�hnten Datensatzes. In zwei neuen Ans�tzen wird das Alignment zwischen den Phonemen des Textes und mit den neuen akustischen Modellen erzeugten Posteriorgrammen berechnet, entweder mit Dynamic Time Warping oder mit der Levenshtein-Distanz. Die letztgenannte Methode erzielte bei der \textit{MIREX} 2017 Lyrics-to-Audio Alignment Challenge die besten Ergebnisse, und erreicht Accuracies von $0.54$ f�r kurze Segmente und $0.94$ f�r ganze Songs. Au�erdem wird eine praktische Anwendung f�r die automatische Schimpfwortsuche vorgestellt.
\end{abstract}

\end{otherlanguage}

%\cleardoublepage
%\ihead[]{Acknowledgements}
%\ohead[]{\pagemark}
\newpage

\begin{verbatim}




 





\end{verbatim}

\textbf{\emph{Acknowledgements}}
\bigskip
\\



\footnotesize
\tableofcontents

% Die einzelnen Kapitel
\cleardoublepage % Seitenumbruch erzwingen vor nderung des Nummerierungsstils
\normalsize

\pagenumbering{arabic} % Nummerierung der Seiten ab hier: 1, 2, 3, 4...
\chapter{Introduction} \label{chap:introduction}
\section{Motivation}\label{sec:intro_motivation}
%mir - no lyrics recognition
%asr - no singing
%similar
%applications

Ever since the widespread introduction of digital formats for music, professional and personal music collections have grown exponentially. Efficient search algorithms are necessary for managing these huge collections. In the past \~15 years, many interesting technologies have been developed to make it easier for users to efficiently search these collections by certain semantic criteria, such as tempo, mood, genre, instruments, etc. This field of research is called Music Information Retrieval (MIR) \cite{incollection:mir}. One characteristic that has not received much research attention yet is the lyrical content of songs, even though this information is useful for many practical applications, and could aid other MIR tasks.\\

On the other hand, Automatic Speech Recognition (ASR) has been an active field of research for more than 30 years now and encompasses a large variety of research topics. However, speech recognition algorithms have so far only rarely been adapted to singing. One of the reasons for this seems to be that most of these tasks get harder when using singing because singing data has different characteristics, which are also often more varied than in pure speech \cite{loscos}. For example, the typical fundamental frequency for women in speech is between $165$ and $200 Hz$, while in singing it can reach more than $1000 Hz$. Other differences include harmonics, durations, pronunciation, and vibrato.\\

Generally, both fields of research are strongly related and utilize many of the same approaches and technologies. This overlap, however, has not been explored thoroughly. This work aims to look at this relation more closely, and to apply and adapt ASR technologies to singing.\\

The possibilities for practical use of such technologies are manifold. These applications may include, but are not limited to:
\begin{description}
\item[Direct search of music based on its lyrical content] Potential users could search for songs by their language, lyrical phrases, or keywords. This is useful for such applications as language learning, finding songs on certain topics, advertisement etc.
\item[Improvement of similarity search and playlist generation] Similarity dimensions could include the sung language, keywords, or topics.
 \item[Improvement of regional classification] As described in \cite{kruspe11}, human subjects tend to rely on the language to determine the region of origin of a musical piece. This is not taken into account by current regional classification systems.
 \item[Improvement of genre classification] Similar to regional classification, certain musical genres are closely connected to a single singing language, or to certain keywords. Considering the ``glass ceiling'' of approximately 80\% for many classification tasks in music information retrieval \cite{glass_ceiling}, new hybrid approaches are necessary to improve them.
 \item[Improvement of mood detection] Certain words are indicatory of specific moods. In this way, mood detection in music could be expanded with an additional dimension.
 \item[Lyrics alignment for karaoke] Automatically aligned lyrics could be used in karaoke systems to enable users to sing any song they want to.
\item[Lyrics retrieval from databases] Textual lyrics can be retrieved from a known database with just a short sung recording as the input. This is, once again, useful for karaoke. 
\item[Lyrics identification or fingerprinting] It is possible to compute compressed representations of detected lyrics. These could be used to aid in fingerprinting and audio identification (query-by-humming) technologies by utilizing lyrics information in addition to the melodic and harmonic characteristics exploited so far. 
\item[Cover song detection by lyrics] In the same vein, alternative or auxiliary technologies for cover song detection by lyrics analysis are possible.
\item[Lyrics transcription] Given an audio recording, it will eventually be possible to automatically transcribe the full lyrics for users.
\item[Singing generation] Similar to recent approaches for speech synthesis \cite{wavenet}, ASR technologies for singing could be used to automatically generate singing audio.
\end{description}


\section{Research objectives}
For this thesis, five topics in speech recognition for singing were researched:

\paragraph{Phoneme recognition}
Phoneme recognition describes the task of determining the sung sounds (phonemes) occurring in a an audio recording. This forms the basis for many other tasks; first and foremost, lyrics transcription, but also almost all other tasks in this work. Phoneme recognition tends to be the bottleneck component in any systems for ASR in singing. Inaccurate results at this step will lead to inaccurate results in the following ones.\\
As will be shown, phoneme recognition in singing has so far been performed with models trained on speech; these are, of course, not optimal. The reason why models have so far not been trained on singing is the lack of available training data for this task. This thesis presents ways around this problem. Specifically, acoustic models are trained on speech data that has been made more ``song-like'', and on singing data with automatically generated phoneme annotations. These new models demonstrate improvements on this and all the following tasks.

\paragraph{Language identification}
Language identification is the task of detecting the language a sung recording is being performed in. This has many practical applications, as described above: The results could be employed to directly search for music in certain languages (e.g. for language learning or for advertisements), to improve similarity search algorithms, or to support regional and genre classification.\\
There are very few publications dealing with singing language identification so far. In this thesis, a state-of-the-art approach from the field of Automatic Speech Recognition (ASR) is applied to the problem, and a completely new one based on phoneme statistics is presented.

\paragraph{Keyword spotting}
During keyword spotting, a set of singing recordings is searched for a specific keyword. Just like language identification, there are practical motivations for this. Keyword-based search systems are useful for finding songs on certain topics, for playlist generation, similarity search, genre classification, or for mood detection.\\
Once again, there are very few published approaches for this task. This thesis presents the first approach for English-language keyword spotting of arbitrary keywords without side information (like the musical score or sung samples of the keyword). Additionally, a new method for integrating knowledge about plausible phoneme durations is described.

\paragraph{Lyrics-to-audio alignment}
Using an audio recording and its known textual lyrics, lyrics-to-audio alignment methods are able to determine where each phrase, word, or phoneme occurs in time. In contrast to the other tasks, this topic is already relatively well-researched. It is a sought-after technology for karaoke applications, or for supporting other speech recognition tasks (e.g. keyword spotting becomes much easier when the textual lyrics are available).\\
In this thesis, classic HMM-based alignment is first used as an auxiliary technology to create a new training data set for phoneme recognition. Then, two new methods are presented: One based on Dynamic Time Warping (DTW) on the results of the phoneme recognition, and one based on Levenshtein distance calculation on phoneme sequences.

\paragraph{Lyrics retrieval}
Another research topic that has not received much attention so far, lyrics retrieval is the task of finding the correct textual lyrics (and consequently the correct song) in a database given a sung query. This is, again, useful for karaoke systems or generally for voice-based search.\\
This work presents new approaches for this task that utilize the same technologies as the audio alignment. Compared to the state of the art, these are the first systems that do not require melody information in addition to the lyrics, and also work directly on the detected phonemes without a language modeling step required (which is, in effect, then a text search on the detected phrases).

\section{Thesis structure}
The thesis is structured into nine chapters:
\begin{description}
\item[1 Introduction] This chapter. Motivates the work and describes the research goals.
\item[2 Technical background] Describes the various algorithms for feature extraction, machine learning, and distance calculation used throughout this work. Also explains the general system structure of the developed approaches as well as their evaluation.
\item[3 State of the art] Summarizes existing methods for solving the mentioned research objectives.
\item[4 Data sets] An overview of the various data sets used for training and testing the developed approaches.
\item[5 Singing phoneme recognition] Describes the approaches developed for the phoneme recognition task and their evaluation results.
\item[6 Sung language identification] Presents the developed methods for language identification and their evaluation results.
\item[7 Sung keyword spotting] Explains the keyword spotting algorithms and their evaluation results.
\item[8 Lyrics retrieval and alignment] Describes the developed systems for lyrics alignment and retrieval and their evaluation results.
\item[9 Conclusion] Summarizes the work, points out the major contributions, and suggests future research directions.
\end{description}





%ozeki??

\chapter{State of the art}	\label{chap:sota}
This chapter presents an overview over the various published approaches for the tasks considered in this work. One key issue is comparability: As in many MIR tasks, data sets are not publicly available and vary widely, making comparison impossible. As the following sections show, there is also disagreement about evaluation measures in all of these tasks.\\
For these reasons, new, reproducible data sets were created for this work (presented in the next chapter). As described in the previous chapter, the most frequently used measures were employed for evaluation.

\section{From speech to singing} \label{sec:sota_speechtosinging}
Singing presents a number of challenges for speech recognition when compared to pure speech \cite{loscos}\cite{goto_alignment}\cite{kruspe_kws1}. The following factors make speech recognition on singing more difficult than on speech, and necessitate the adaption of existing algorithms.
\begin{description}
 \item[Larger pitch fluctuations] A singing voice varies its pitch to a much higher degree than a speaking voice. It often also has very different spectral properties.
 \item[Larger changes in loudness] In addition to pitch, loudness also fluctuates much more in singing than in speech.
 \item[Higher pronunciation variation] The musical context causes singers to pronounce certain sounds and words differently than if they were speaking them.
 \item[Larger time variations] In singing, sounds are often prolonged for a certain amount of time to fit them to the music. Conversely, they can also be shortened or left out completely.\\
 In order to research this effect more closely, a small experiment was performed on a speech corpus and on a singing data set (\textit{TIMIT} and \textit{ACAP}, see chapter \ref{chap:datasets}): The standard deviations for all the phonemes in each data set were calculated. The result is shown in figure \ref{fig:phoneme_stats}, confirming that the variation in singing is much higher. This is particularly true for vowels.
 \item[Different vocabulary] In musical lyrics, words and phrases often differ from normal conversational script. Certain words and phrases have different probabilities (e.g. a higher focus on emotional topics in singing).
 \item[Background music] This is the biggest interfering factor with polyphonic recordings. Harmonic and percussive instruments add a big amount of spectral components to the signal, which lead to confusion in speech recognition algorithms. Ideally, these components should be removed or suppressed in a precursory step. This could be achieved, for example, by employing source separation algorithms. However, such algorithms add additional artifacts to the signal, and may not even be sufficient for this purpose at the current state of research.\\
 Vocal activity detection (VAD) could be used as a non-invasive first step in order to discard segments of songs that do not contain singing voices. However, such algorithms often make mistakes in the same cases that are problematic for speech recognition algorithms (e.g. instrumental solos \cite{schlueter2016_ismir}).\\
 For these reasons, most of the experiments in this work were performed on unaccompanied singing. The integration of the mentioned pre-processing algorithms would be a very interesting next step of research.\\
The lyrics-to-singing alignment algorithms presented in chapter \ref{chap:retrieval_alignment} are an exception. Those were also tested on polyphonic music, and the algorithms appear to be largely robust to these influences.
 \end{description}
 %Mehr Beispiele aus Goto-Kapitel
 %Sundberg!
 % Vowels vs. Consonants (1. Mesaros-Paper + recognition of phonemes and words...)

\begin{figure*}
	\begin{center}
		\includegraphics[width=1\textwidth]{images/phoneme_stats.png}
		\caption{Standard deviations of phoneme durations in the \textit{TIMIT} and \textit{ACAP} data sets.}
		\label{fig:phoneme_stats}
	\end{center}
\end{figure*}

%describe tasks here??

\section{Phoneme recognition} \label{sec:sota_phonerec}
%\subsection{Phoneme recognition in speech}
%\subsection{Phoneme recognition in singing}
Due to the factors mentioned above in section \ref{sec:sota_speechtosinging}, phoneme recognition on singing is more difficult than on clean speech. It has only been a topic of research for a few years, and there are few publications.\\
One of the earliest systems was presented by Wang et al. in 2003 \cite{WangLC03}. Acoustic modeling is performed with triphone HMMs trained on read speech in Taiwanese and Mandarin. The language model is completely restricted to lines of lyrics in the test dataset. Testing is performed on 925 unaccompanied sung phrases in these language. Due to the highly specific language model, the word error rate is just $0.07$.\\

Hosoya et al. employ a similarly classical approach from ASR that employs monophone HMMs also trained on read speech for acoustic modeling \cite{Hosoya2005} (2005). These models are adapted to singing voices using the Maximum Likelihood Linear Regression (MLLR) technique \cite{mllr}. Language modeling is performed with a Finite State Automaton (FSA) specific to the Japanese language, making it more flexible than the previous system.\\
The system is tested on five-word unaccompanied phrases, while the adaptation is performed on 127 choruses performed by different singers. The Word Error Rate is $0.36$ without the adaptation, and $0.27$ after adaptation.\\

In 2007, Gruhne et al. presented a classical approach that employs feature extraction and various machine learning algorithms to classify singing into 15 phoneme classes \cite{Gruhne2007} \cite{Gruhne2007a}. The specialty of this approach lies in the pre-processing: At first, fundamental frequency estimation is performed on the audio input, using a Multi-Resolution Fast Fourier Transform (MRFFT) \cite{inproceedings:dressler}. Based on the estimated fundamental frequency, the harmonic partials are retrieved from the spectrogram. Then, a sinusoidal re-synthesis is performed, using only the detected fundamental frequency and partials. Feature extraction is then performed on this re-synthesis instead of the original audio. Extracted features include MFCCs, PLPs, Linear Predictive Coding features (LPCs), and Warped Linear Predictive Coding features (WLPCs \cite{lpc}). MLP, GMM, and SVM models are trained on the resulting feature vectors. The re-synthesis idea comes from a singer identification approach by Fujihara \cite{fujihara_identification}.\\
The approach is tested on more than 2000 separate, manually annotated phoneme instances from polyphonic recordings. Only one feature vector per phoneme instance is calculated. Using SVM models, $56\%$ of the tested instances were classified correctly into one of the 15 classes. This is significantly better than the best result without the re-synthesis step ($34\%$).\\
In \cite{szepannek} (2010), the approach is expanded by testing a larger set of perceptually motivated features, and more classifiers. No significant improvements are found when using more intricate features, and the best-performing classifier remains SVM.\\

Fujihara et al. described an approach based on spectral analysis in 2009 \cite{fujihara_phonemes}. The underlying idea is that spectra of polyphonic music can be viewed as the weighted sum of two types of spectra: One for the singing voice, and one for the background music. This approach then models these two spectra as probabilistic spectral templates. The singing voice is modeled by multiplying a vocal envelope template, which represents the spectral structure of the singing voice, with a harmonic filter, which represents the harmonic structure of the produced sound itself. This is analogous to the source-filter model of speech production \cite{Fant1981}. For recognizing vowels, five such harmonic filters are prepared (\texttt{/a/ - /e/ - /i/ - /o/ - /u/}). Vocal envelope templates are trained on voice-only recordings, separated by gender. Templates for background music are trained on instrumental tracks. In order to recognize vowels, the probabilities for each of the five harmonic templates are estimated. As a side product, the algorithm also estimates the fundamental frequency of the singing voice.\\
As described, the phoneme models are gender-specific and only model five vowels, but also work for singing with instrumental accompaniment. The approach is tested on 10 Japanese-language songs. The best result is $65\%$ correctly classified frames, compared to the $56\%$ with the previous approach by this team, based on GMMs. \\

In 2009, Mesaros et al. also picked Hosoya's approach back up by using MFCC features and GMM-HMMs for acoustic modeling \cite{Mesaros2009}, and adapting the models for singing voices. These models are trained on the CMU ARCTIC speech corpus\footnote{\url{http://festvox.org/cmu_arctic/}}. Then, different MLLR techniques for adapting the models to singing voices are tested \cite{mllr}.\\
The adaptation and test corpus consists of 49 voice-only fragments from 12 pop songs with durations between 20 and 30 seconds. The best results are achieved when both the means and variances of the Gaussians are transformed with MLLR. The results improved slightly when not just a single transform was used for all phonemes, but when they were grouped into base classes beforehand, each receiving individual transformation parameters. The best result is around $0.79$ Phoneme Error Rate on the test set.\\
In \cite{Mesaros2010} and \cite{Mesaros2011}, language modeling is added to the presented approach. Phoneme-level language models are trained on the CMU ARCTIC corpus as unigrams, bigrams, and trigrams, while word-level bigram and trigram models are trained on actual song lyrics in order to match the application case. The output from the acoustic models is then refined using these language models. The approach is tested on the clean singing corpus mentioned above, and on 100 manually selected fragments of 17 polyphonic pop songs. To facilitate recognition on polyphonic music, a vocal separation algorithm is introduced \cite{virtanen_separation}.\\
Using phoneme-level language modeling, the Phoneme Error Rate on clean singing is reduced to $0.7$. On polyphonic music, it is $0.81$. For the word recognition approach, the word error rate is $0.88$ on clean singing, and $0.94$ on the polyphonic tracks.\\
A more detailed voice adaptation strategy is tested in \cite{mesaros2}. Instead of adapting the acoustic models with mixed-gender singing data, they are adapted gender-wise, or to specific singers. With the gender-specific adaptations, the average Phoneme Error Rate on clean singing is lowered to $0.81$ without language modeling, and $0.67$ with language modeling. Singer-specific adaptation does not improve the results, probably because of the very small amount of adaptation data in this case.\\
%mvcivar, hosoya
In \cite{McVicar2014} (2014), McVicar et al. build on a very similar baseline system, but also exploit repetitions of choruses to improve transcription accuracy. This has been done for other MIR tasks, such as chord recognition, beat tracking, and source separation. They propose three different strategies for combining individual results: Feature averaging, selection of the chorus instance with the highest likelihood, and combination using the Recogniser Output Voting Error Reduction (ROVER) algorithm \cite{rover}. They also employ three different language models, two of which were matched to the test songs (and therefore not representative for general application). 20 unaccompanied, English-language songs from the RWC database \cite{rwc} were used for testing; chorus sections were selected manually. The best-instance selection and the ROVER strategies improve results significantly; with the ROVER approach and a general-purpose language model, the Phoneme Error Rate is at $0.74$ (versus $0.76$ in the baseline experiment), while the Word Error Rate is improved from $0.97$ to $0.9$. Interestingly, cases with a low baseline result benefit the most from exploiting repetition information.\\

The final system was proposed by Hansen in 2012 \cite{jens}. It also employs a classical approach consisting of a feature extraction step and a model training step. Extracted features are MFCCs and TRAP features. Then, MLPs are trained separately on both feature sets. As mentioned in section \ref{subsec:tech_trap}, each feature models different properties of the considered phonemes: Short-term MFCCs are good at modeling the pitch-independent properties of stationary sounds, such as sonorants and fricatives. On the flip-side, TRAP features are able to model temporal developments in the spectrum, forming better representations for sounds like plosives or affricates.\\
The results of both MLP classifiers are combined via a fusion classifier, also an MLP. Then, Viterbi decoding is performed on its output.\\
The approach is trained and tested on a data set of 13 vocal tracks of pop songs, which were manually annotated with a set of 27 phonemes. The combined system achieves a recall of $0.48$, compared to $0.45$ and $0.42$ for the individual MFCC and TRAP classifiers respectively. This confirms the assumption that the two features complement each other. The phoneme-wise results further corroborate this.

\section{Lyrics-to-audio alignment}
%\subsection{Forced alignment in speech}
%\subsection{Forced alignment in singing}
%mesaros
%Three techniques for im- proving automatic synchronization between music and lyrics: Fricative detection, filler model, and novel feature vectors for vocal activity detection
In contrast to the other tasks discussed in this chapter, the task of lyrics-to-audio alignment has been the focus of many more publications. A comprehensive overview until 2012 is given in \cite{goto_alignment}.\\
A first approach was presented in 1999 by Loscos et al. \cite{loscos}. The standard forced alignment approach from speech recognition is adapted to singing. MFCCs are extracted first, and then a left-to-right HMM is employed to perform alignment via Viterbi decoding. Some modifications are made to the Viterbi algorithm to allow for low-delay alignment. The approach is trained and tested on a very small (22 minutes) database of unaccompanied singing, but no quantitative results are given.\\
The first attempt to synchronize lyrics to polyphonic recordings was made by Wang et al. in 2004 \cite{Wang2004}. They propose a system, named ``LyricAlly'', to provide line-level alignments for karaoke applications. Their approach is heavily based on musical structure analysis. First, the hierarchical rhythm structure of the song is estimated. The result is combined with an analysis of the chords and then used to split the song into sections by applying a chorus detection algorithm. Second, Vocal Activity Detection (VAD) using HMMs is performed on each section. Then, sections of the text lyrics are assigned to the detected sections (e.g. verses, choruses). In the next step, the algorithm determines whether the individual lines of the lyrics match up with the vocal sections detected by the VAD step. If they do not, grouping or partitioning is performed. This is based on the assumption that lyrics match up to rhythmic bars as determined by the hierarchical rhythm analysis. The expected duration of each section and line is estimated using Gaussian distributions of phoneme durations from a singing data set. In this manner, lines of text are aligned to the detected vocal segments. The approach is tested on 20 manually annotated pop songs. On the line level, the average error is $0.58$ seconds for the starting points and $-0.48$ seconds for the durations. The system components are analyzed in more detail in \cite{lyrically}.\\
In 2006, the same team presented an approach that also performs rhythm and bar analysis to facilitate syllable-level alignment \cite{Iskandar:2006}. For the phoneme recognition step, an acoustic model is trained on speech data and adapted to singing using the previously mentioned 20 songs. The possible syllable positions in the alignment step are constrained to the note segments detected in the rhythm analysis step. Due to annotator disagreement on the syllable level, the evaluation is performed on the word level. On three example songs, the average synchronization error rate is $0.19$ when allowing for a tolerance of 1/4 bar.\\  

Sasou et al. presented a signal parameter estimation method for singing employing an auto-regressive HMM (AR-HMM) in 2005 \cite{Sasou2005AnAN}. This method is particularly suited for modeling high-pitched signals, which is important for singing voices and usually not a focus in speech processing techniques. Models trained on speech are adapted to singing using MLLR. For evaluation, the method is applied to the task of lyrics-to-audio alignment and tested on 12 Japanese-language songs. For each song, a specific language model is prepared. The correct word rate is $0.79$.\\

Chen et al. presented an approach based on MFCC features and Viterbi alignment in 2006 \cite{popular_synchronization}. Vocal Activity Detection is performed as a pre-processing step, and then GMM-HMMs are used for Viterbi alignment between the audio and the lyrics. Once again, MLLR is used to adapt the acoustic models to singing. In addition, the grammar is specifically tailored to the lyrics. On a data set of Chinese songs by three singers, a boundary accuracy of $0.76$ is obtained on the syllable level.\\

A similar approach which does not require in-depth music analysis was presented by Fujihara et al. in 2006 \cite{fujihara_alignment}. Once again, a straightforward Viterbi alignment method from speech recognition is refined by introducing three singing-specific pre-processing steps: Accompaniment sound reduction, Vocal Activity Detection, and phoneme model adaptation.\\
For accompaniment reduction, the previously mentioned harmonic re-synthesis algorithm from \cite{fujihara_identification} is used. For Vocal Activity Detection, a HMM is trained on a small set of unaccompanied singing using LPC-derived MFCCs and fundamental frequency ($F_0$) differences as features. The HMM can be parameterized to control the rejection rate. For the phoneme model adaptation, three consecutive steps are tested: Adaptation to a clean singing voice, adaptation to a singing voice segregated with the accompaniment reduction method, and on-the-fly adaptation to a specific singer. MFCC features are used for the Viterbi alignment, which is performed on the vowels and syllabic nasals (\texttt{/m/, /n/, /l/}) only.\\
Ten Japanese pop songs were used for testing. Evaluation was done one the phrase level by calculating the proportion of the duration of correctly aligned sections to the total duration of the song. For eight of the ten songs, this proportion was $0.9$ or higher when using the complete system, which the authors judge as satisfactory. Generally, the results are lower for performances by female singers, possibly because of the higher $F_0$s. These performances also benefit the most from the Vocal Activity Detection step, even though its performance is also somewhat worse for female singing. All three levels of phoneme model adaptations contribute to the success of the approach.\\
In 2008, the authors improved upon this system with three modifications: Fricative detection, filler models, and new features for the Vocal Activity Detection step \cite{fujihara}. Fricative detection is introduced because the previous system was only based on vowels and nasals, due to the fact that the harmonic re-synthesis discards other consonants. In the new system, fricatives are detected before this step and then retained for the alignment (stops are not used because they are too short).\\
The filler model is employed because singers sometimes add extraneous lyrics (like ``la la la'' or ``yeah'') to their performances.\\
As mentioned above, Vocal Activity Detection does not work as well for female performances because of inaccuracies in the spectral envelope estimation in high pitch regions. For this reason, the features are replaced in the new version by comparing the power of the harmonic components directly with those of similar $F_0$ regions.\\
The approach is again evaluated on ten Japanese pop songs. The original system produces an average accuracy of $0.81$, which is raised to $0.85$ with the new improvements.\\
In \cite{FujiharaGOO11}, the whole system is presented succinctly and evaluated in more detail. Additionally, integration into a music playback interface is described.\\

Mauch et al. augmented the same approach in 2010 by using chord labels, which are often available in combination with the lyrics on the internet \cite{mauch_alignment2010}\cite{mauch_alignment2}. Chords usually have longer durations than individual phonemes, and are therefore easier to detect. In this way, they provide a coarse alignment, which can be used to simplify the shorter-scale phoneme-level alignment. A chroma-based approach is used to estimate the chords. Information about the chord alignments is directly integrated into the HMM used for alignment.\\
In \cite{mauch_alignment2}, a large range of parameterizations is tested on 20 English-language pop songs. The highest accuracy for the baseline approach (without chord information) is $0.46$. Using chord position information, this rises to $0.88$. Interestingly, Vocal Activity Detection improves the result when not using chords, but decreases it for the version with chord alignments, possibly because the coarse segmentation it provides is already covered by the chord detection in the second case. The method is also able to cope with incomplete chord transcriptions while still producing satisfactory results.\\

In 2007, Wong et al. presented a specialized approach for Cantonese singing that does not require phoneme recognition \cite{WongSW07}. Since Cantonese is a tonal language, prosodic information can be inferred from the lyrics. This is done by estimating relative pitch and timing of the syllables by using linguistic rules. On the other side, a vocal enhancement algorithm is applied to the input signal, and its pitches and onsets are calculated. Then, both sets of features are aligned using DTW.\\ 
14 polyphonic songs were used for evaluation. The approach reaches an average (duration) accuracy of $0.75$.\\

Lee et al. also follow an approach without phoneme recognition \cite{LeeC08} (2008). It is purely based on structural analysis of the song recording, which is performed by calculating a self-similarity matrix and using it for segmentation. The algorithm takes structural a-priori knowledge into account, e.g. the fact that choruses usually occur most frequently and do not differ much internally. Lyrics segments are annotated by hand, splitting them up into paragraphs and labeling them with structural part tags (``intro'', ``verse'', ``chorus'', and ``bridge''). Then, Dynamic Programming is performed to match the lyrics paragraphs to the detected musical segments. A Vocal Activity Detection step is also introduced. Testing the approach on 15 English-language pop songs with 174 lyrics paragraphs in total, they obtain an average displacement error of $3.5$ seconds.\\ %segmentation-based

Mesaros et al. also present an alignment approach that makes use of their phoneme recognition approach described above in \ref{sec:sota_phonerec} \cite{mesaros_alignment}, adding a harmonic re-synthesis step for vocal separation. Based on these models, they employ Viterbi alignment and obtain an average displacement error of $1.4$ seconds for line-wise alignment ($0.12$ seconds when not using absolute differences). The test set consists of 17 English-language pop songs. They identify mistakes in the vocal separation step as the main source of error.\\ %mention other mesaros papers

Two specialized approaches were presented in the past two years. The first one is part of a score-following algorithm by Gong et al. \cite{gong_alignment}. Here, vowels are used to aid alignment of the musical score to the recording, assuming the score contains the lyrics. This is done by training vowel templates on a small set of sung vowels. Spectral envelopes are used as features. Two different strategies for fusing the vowel and melody information are tested (``early'' and ``late''), as well as singer-specific adaptation of the templates via Maximum A-Posteriori (MAP) estimation. The training set consists of 160 vowel instances per singer, the test set of 8 full unaccompanied French-language songs per singer.  The average displacement error is around 68ms for both singer-specific and -adapted models (best strategy).\\

Finally, Dzhambazov et al. presented a method that integrates knowledge of note onsets into the alignment algorithm \cite{dzhambazov_alignment}. Pitch extraction and note segmentation are performed in parallel with phoneme recognition via HMMs, and both results are refined with a transition model. A variable time HMM (VTHMM) is used to model the rules for phoneme transitions at note onsets.\\
On a test dataset of 12 unaccompanied Turkish-language Makam performances, the method achieves an alignment accuracy of $0.76$. For polyphonic recordings (usually with accompaniment by one or more string instruments), a vocal re-synthesis step is introduced. The average accuracy in this case is $0.65$.



\section{Lyrics retrieval}
As described in \ref{sec:sota_phonerec}, Hosoya et al. developed a system for phoneme recognition, which they also apply to lyrics retrieval \cite{Hosoya2005}. On a dataset of 238 children's songs, they obtain a retrieval rate of $0.86$ for the Top 1 result, and of $0.91$ for the Top 10 results. In \cite{suzuki06} and \cite{suzuki07}, more experiments are conducted. As a starting point, the number of words in the queries is fixed at 5, resulting in a retrieval rate of $0.9$ (Top 1 result). Then, a melody recognition is used to verify the matches proposed by the speech recognition step, raising the retrieval rate to $0.93$. The influence of the number of words in the query is also evaluated, confirming that retrieval becomes easier the longer the query is. However, even at a length of just three words, the retrieval rate is $0.87$ (vs. $0.81$ without melody verification).\\

Similarly, Wang et al. presented a query-by-singing system in 2010 \cite{Wang2010}. The difference here is that melody and lyrics information are weighted equally in the distance calculation. Lyrics are recognized with a bigram HMM model trained on speech. The results are interpreted as syllables. A syllable similarity matrix is employed for calculating phoneme variety in the query, which is used for singing vs. humming discrimination. Assuming that only the beginning of each song is used as the starting point for queries, the first 30 syllables of each song are transformed into an Finite State Machine (FSM) language model and used for scoring queries against each song in the database. The algorithm is tested on a database of 2154 Mandarin-language songs, of which 23 were annotated and the remainder are used as ``noise'' songs. On the Top 1 result, a retrieval rate of $0.91$ is achieved for the system combining melody and lyrics information, compared to $0.88$ for the melody-only system.\\

%RETRIEVAL: Mesaros!!!
As described in \ref{sec:sota_phonerec}, Mesaros et al. developed a sophisticated system for phoneme and word recognition in singing. In \cite{mesaros1}, \cite{mesaros2}, and \cite{Mesaros2011}, they also describe how this system can be used for lyrics retrieval. This is the only purely lyrics-based system in literature. Retrieval is performed by recognizing words in queries with the full system, including language modeling, and then ranking each lyrics segment by the number of matching words (bag-of-words approach). The lyrics database is constructed from 149 song segments (lasting between 9 and 40 seconds in the corresponding recordings). Recordings of 49 of these segments are used as queries to test the system. The Top 1 retrieval rate is $0.57$ ($0.71$ for the Top 10).


\section{Language identification}
A first approach for language identification in singing was proposed by Tsai and Wang in 2004 \cite{tsai_wang}. At its core, the algorithm is similar to Parallel Phone Recognition followed by Language Modeling (PPRLM). However, instead of full phoneme modeling, they employ an unsupervised clustering algorithm to the input feature data and tokenize the results to form language-specific codebooks (plus one for background music). Following this, the results from each codebook are run through matching language models to determine the likelihood that the segment was performed in this language. Prior to the whole process, Vocal Activity Detection is performed. This is done by training GMMs on segments of each language, and on non-vocal segments. MFCCs are used as features.\\
The approach is tested on 112 English- and Mandarin-language polyphonic songs each, with 32 of them being the same songs performed in both languages. A classification accuracy of $0.8$ is achieved on the non-overlapping songs. On the overlapping songs, the accuracy is only $0.7$, suggesting some influence of the musical material (as opposed to the actual language characteristics). Misclassifications occur more frequently on the English-language songs, possibly because of accents of Chinese singers performing in English, and because of louder background music.\\

A second, simpler approach was presented by Schwenninger et al. in 2006 \cite{schwenninger}. They also extract MFCC features, and then use these to directly train statistical models for each language. Three different pre-processing strategies are tested: Vocal Activity Detection, distortion reduction, and azimuth discrimination. Vocal Activity Detection (or vocal/non-vocal segmentation) is performed by thresholding the energy in high-frequency bands as an indicator for voice presence over 1-second windows. This leaves a relatively small amount of material per song. Distortion reduction is employed to discard strong drum and bass frames where the vocal spectrum is masked by using a Mel-scale approach. Finally, azimuth discrimination attempts to detect and isolate singing voices panned to the center of the stereo scene.\\
The approach is tested on three small data sets of speech, unaccompanied singing, and polyphonic music. Without pre-processing steps, the accuracy is $0.84$, $0.68$, and $0.64$ respectively, highlighting the increased difficulty of language identification on singing versus speech, and on polyphonic music versus pure vocals. On the polyphonic corpus, the pre-processing steps do not improve the result.\\

In 2011, Mehrabani and Hansen presented a full PPRLM approach for sung language identification. MFCC features are run through phoneme recognizers for Hindi, German, and Mandarin; then, the results are scored by individual language models for each considered language. In addition, a second system is employed which uses prosodic instead of phonetic tokenization. This is done by modeling pitch contours with Legendre polynomials, and then quantizing these vectors with previously trained GMMs. The results are then again used as inputs to language models.\\
The approach is trained and tested on a corpus containing 12 hours of unaccompanied singing and speech in Mandarin, Hindi, and Farsi. The average accuracy for singing is $0.78$ and $0.43$ for the phoneme- and prosody-based systems respectively, and $0.83$ for a combination of both.\\

Also in 2011, Chandrasekhar et al. presented a very interesting approach for language identification on music videos, analyzing both audio and video features \cite {chandrasekhar}. On the audio side, the spectrogram, volume, MFCCs, and perceptually motivated Stabilized Auditory Images (SAI) are used as inputs. One-vs-all SVMs are trained for each language. The approach is trained and tested on 25,000 music videos in 25 languages. Using audio features only, the accuracy is $0.45$; combined with video features, it rises to $0.48$. It is interesting to note that European languages seem to achieve much lower accuracies than Asian and Arabic ones. English, French, German, Spanish and Italian rank below $0.4$, while languages like Nepali, Arabic, and Pashto achieve accuracies above $0.6$. It is possible that the language characteristics of European languages make them harder to discriminate (especially against each other) than others.

\section{Keyword spotting}
%\subsection{Keyword spotting in speech}
%Igor Sz? oke, Petr Schwarz, Pavel Matejka, Luk?as Bur-get, Martin Karafi? at, Michal Fapso, and Jan Cernock`y.Comparison of keyword spotting approaches for infor- mal continuous speech. In Interspeech, pages 633?636, 2005.
%\subsection{Keyword spotting in singing}
%SPEECH HERE
%pedro!!
Keyword spotting in singing was first attempted in 2008 by Fujihara et al. \cite{hyperlinking_lyrics}. Their method starts with a phoneme recognition step, which is once again based on the vocal re-synthesis method described in \cite{fujihara_identification}. MFCCs and power features are extracted from the re-synthesized singing and used as inputs to a phoneme model, similar to Gruhne's phoneme recognition approach mentioned above in \ref{sec:sota_phonerec}.  Three phoneme models are compared: One trained on pure speech and adapted with a small set of singing recordings, one adapted with all recordings, and one trained directly on singing. Viterbi decoding is then performed using keyword-filler HMMs (see \ref{subsec:tech_kws}) to detect candidate segments where keywords may occur. These segments are then re-scored through the filler HMM to verify the occurrence.\\
The method is tested on 79 unaccompanied Japanese-language songs from the RWC database \cite{rwc} with keywords containing at least 10 phonemes. The Phoneme Error Rate is $0.73$ for the acoustic models trained on speech, $0.67$ for the adapted models, and $0.49$ for the models trained on singing (it should be mentioned that the same songs were used for training and testing, although a cross-validation experiment shows that the effect is negligible). The employed evaluation measure is ``link success rate'', describing the percentage of detected phrases that were linked correctly to other occurrences of the phrase in the data set. In that sense, it is a sort of accuracy measure. The link success rate for detecting the keywords is $0.3$. The authors show that the result depends highly on the number of phonemes in the considered keyword, with longer keywords being easier to detect.\\

In 2012, Mercado et al. presented an approach to keyword spotting in singing based on a different principle: DTW between a sung query and the requested phrase in the song recording. In particular, Statistical Sub-Sequence DTW is the algorithm employed for this purpose. MFCCs are used as feature inputs, then the costs of the warping paths are calculated from all possible starting points to obtain candidate segments, which are then further refined to find the most likely position.\\
The approach is tested on a set of vocal tracks of 19 pop songs (see section \ref{subsec:data_hansen}) as the references, and recordings of phrases sung by amateur singers as the queries, but no quantitative results are given. The disadvantage of this approach lies in the necessity for audio recordings of the key phrases, which need to have at least similar timing and pitch as the reference phrases.\\

Finally, Dzhambazov et al. developed a score-aided approach to keyword spotting in 2015 \cite{dzhambazov_ismir}. A user needs to select a keyword phrase and a single recording in which this phrase occurs. The keyword is then modeled acoustically by concatenating recordings of the constituent phonemes (so-called acoustic keyword spotting). Similar to Mercado's approach, Sub-Sequence DTW is performed between the acoustic template and all starting positions in the reference recording to obtain candidate segments. These segments are then refined by aligning the phonemes to the score in these positions to model their durations. This is implemented with Dynamic Bayesian Network HMMs. Then, Viterbi decoding is performed to re-score the candidate segments and obtain the best match.\\
The approach is tested on a small set of unaccompanied Turkish-language recordings of traditional Makam music. The Mean Average Precision (MAP) for the best match is $0.08$ for the DTW approach only, and $0.05$ for the combined approach. For the top-6 results, the MAP is $0.26$ and $0.38$ respectively.



\chapter{Technical Background}	\label{chap:background}
\section{General processing chain}
%audio - pre-processing - feat. extraction - machine learning
% unseen audio - pre-processing - feat. extraction  - run through model - result
\section{Audio features}
%mfcc, sdc, trap; mention filterbank feats...?
\section{Machine learning algorithms}
This section describes the various machine learning algorithms employed throughout this thesis. Gaussian Mixture Models (GMMs), Hidden Markov Models (HMMs), and Support Vector Machines (SVMs) are three traditional approaches that are used as the basis of many new approaches, and were used for several starting experiments. i-Vector processing is a relatively new, more sophisticated approach that bundles several other machine learning techniques.\\
In recent years, Deep Learning has become the standard for machine learning applications \cite{}. This chapter also describes two of those new approaches that were used in this work: Deep Neural Networks (DNNs) and Deep Belief Networks (DBNs).

\subsection{Gaussian Mixture Models}
\subsection{Hidden Markov Models}
\subsection{Support Vector Machines}
\subsection{i-Vector processing}
\subsection{Artificial Neural Networks}
\subsubsection{Deep Neural Networks}
\subsubsection{Deep Belief Networks}
\section{Evaluation}
\subsection{Evaluation of phoneme recognition tasks}
%alignment?
\subsection{Evaluation of language identification tasks}
\subsection{Evaluation of keyword spotting tasks}
\section{Common application systems}
\subsection{Systems for phoneme recognition}
\subsection{Systems for forced alignment}
\subsection{Systems for language identification}
\subsection{Systems for keyword spotting}





%\addtocontents{toc}{\protect\clearpage}
\chapter{Data sets} \label{chap:datasets}
This chapter contains descriptions of all the data sets (or corpora) used over the course of this thesis. They are grouped into speech-only data sets, data sets of unaccompanied (=a-capella) singing, and data sets of full musical pieces with singing (``real-world'' data sets).

\section{Speech data sets}
\subsection{TIMIT}
\textit{TIMIT} is, presumably, the most widely used corpus in speech recognition research \cite{timit}. It was developed in 1993 and consists of 6300 English-language audio recordings of 630 native speakers with annotations on the phoneme, word, and sentence levels. The corpus is split into a training and a test section, with the training section containing 4620 utterances, and the test section containing 1680. Each of those utterances has a duration of a few seconds. The recordings are sampled at $16,000 Hz$ and are mono channel.\\
The phoneme annotations follow a model similar to ARPABET and contain 61 different phonemes \cite{Zue1990}.
%wo benutzt?

\subsection{NIST Language identification corpora}
The National Institute for Standards and Technology (NIST) regularly runs various speech recognition challenges, one of them being the Language Recognition Evaluation (LRE) task, which is offered every two to four years \footnote{\url{https://www.nist.gov/itl/iad/mig/language-recognition}}. To this end, they publish training and evaluation corpora of speech in several languages. They consist of short segments (up to 35 seconds) of free telephone speech by many different speakers. The recordings are mono channel with a sampling rate of $8000 Hz$.\\
The corpus for the 2003 challenge was used in this work for comparison of language identification algorithms against speech data \cite{nist2003lre}. Only the English-, German-, and Spanish-language subsets were chosen, since those languages are covered by the corresponding singing dataset. To balance out the languages, 240 recordings were used for each of them, making up around 1 hour of material.

\subsection{OGI Multi-Language Telephone Speech Corpus}
%aufr�umen!!
In 1992, the Oregon Institute for for Science and Technology (OGI) also published a multilingual corpus of telephone recordings, called the OGI Multi-Language Telephone Speech Corpus, to facilitate multi-language ASR research. Just like the NIST corpora, it has become widely used for speech recognition tasks. Again, the English-, German-, and Spanish-language subsets were used in this work. They each consist of more than 1000 recordings per language of up to $50$ seconds duration, making up a total of about three to five hours. In contrast to the NIST corpus, the recordings are somewhat less ``clean'' with regards to accents and background noise; the sampling rate is also $8000 Hz$ on mono channel.  For experiments on longer recordings, results on these individual utterances were aggregated for each speaker, producing 118 documents per language (354 in sum).\\




%Tabelle nennen!
\begin{table}[htp]
  \caption{{Amounts of data in the three used data sets: Sum duration on top, number of utterances in italics.}}
  \begin{center}
    \begin{tabular}{|c||c|c|c|}\hline
    \textbf{hh:mm:ss} & \multirow{2}{*}{\textbf{NIST2003LRE}} & \multirow{2}{*}{\textbf{OGIMultilang}} & \multirow{2}{*}{\textbf{YTAcap}}\\
    \textbf{\textit{\#Utterances}} & & & \\ \hline \hline 
    \multirow{2}{*}{English} & 00:59:08 & 05:13:17 &  08:04:25 \\
    & \textit{240} & \textit{1912} & \textit{1975} \\ \hline 
    \multirow{2}{*}{German} & 00:59:35 & 02:52:27 & 04:18:57 \\
    & \textit{240} & \textit{1059} & \textit{1052} \\ \hline 
    \multirow{2}{*}{Spanish} & 00:59:44 & 03:05:45 & 07:21:55 \\
    & \textit{240} & \textit{1151} & \textit{1810} \\ \hline 
    \end{tabular}
  \end{center}
  \label{tab:datasets}
\end{table}

\section{Singing data sets}
\subsection{YouTube data set}
As opposed to the speech case, there are no standardized corpora for sung language identification. For the sung language identification experiments, files of unaccompanied singing were therefore extracted from \textit{YouTube}\footnote{\url{http://www.youtube.com}, Last checked: 05/16/13} videos. This was done for three languages: English, German, and Spanish. The corpus consists of between 116 (258min) and 196 (480min) examples per language. These were mostly videos of amateur singers freely performing songs without accompaniment. Therefore, they are of highly varying quality and often contain background noise. The audio was downloaded in the natively provided quality, but then downsampled to $8000 Hz$ and averaged to a mono channel for uniformity and for compatibility with the speech corpora for language identification. Most of the performers contributed only a single song, with just a few providing up to three. This was done to avoid effects where the classifier recognizes the singer's voice instead of the language.\\
Special attention was paid to musical style. Rap, opera singing, and other specific singing styles were excluded. All the songs performed in these videos were pop songs. Different musical styles can have a high impact on language classification results. The author tried to limit this influence as much as possible by choosing recordings of pop music instead of language-specific genres (such as Latin American music).\\
For some experiments, they were split up into segments of 10-20 seconds at silent points (3,156 ``utterances'' in sum).\\
An overview of the amounts of data in the three corpora for language identification is given in Table \ref{tab:datasets}.

\subsection{Hansen's vocal track data set}
This is one of the data sets used for keyword spotting and phoneme recognition. It was first presented in \cite{jens}. It consists of the vocal tracks of 19 commercial English-language pop songs. They are studio quality with some post-processing applied (EQ, compression, reverb). Some of them contain choir singing. These 19 songs are split up into 920 clips that roughly represent lines in the song lyrics. The original audio quality is $44,100 Hz$ on mono channels; for compatibility with models trained on \textit{TIMIT}, they were downsampled to $16,000 Hz$. \\
Twelve of the songs were annotated with time-aligned phonemes. The phoneme set is the one used in CMU Sphinx\footnote{\url{http://cmusphinx.sourceforge.net/}} and TIMIT \cite{timit} and contains 39 phonemes. All of the songs were annotated with word-level transcriptions. This is the only one of the singing data sets that has full manual annotations, which are assumed to be reliable and can be used as ground truth.\\
For comparison, recordings of spoken recitations of all song lyrics were also made. These were all performed by the same speaker (the author).\\
This data set will be referred to as \textit{ACAP}.

\subsection{DAMP data set}
As described, Hansen's data set is very small and therefore not suited to training phoneme models for singing. As a much larger source of unaccompanied singing, the \textit{DAMP} data set, which is freely available from Stanford University\footnote{\url{https://ccrma.stanford.edu/damp/}}\cite{phdthesis:jeffreysmith}, was employed. This data set contains more than 34,000 recordings of amateur singing of full songs with no background music, which were obtained from the \textit{Smule Sing!} karaoke app. Each performance is labeled with metadata such as the gender of the singer, the region of origin, the song title, etc. The singers performed 301 English-language pop songs. The recordings have good sound quality with little background noise, but come from a lot of different recording conditions. They were originally provided in OGG format at a sampling rate of $22,050 Hz$ (mono channel), but were also converted to wave format and downsampled to $16,000 Hz$ for compatibility.\\
No lyrics annotations are available for this data set, but the textual lyrics can be obtained from the \textit{Smule Sing!} website\footnote{\url{http://www.smule.com/songs}}. These are, however, not aligned in any way. Such an alignment was performed automatically on the word and phoneme levels (see section \ref{sec:phonerec_acap}).\\
The selection of songs is not balanced, with performances ranging between 21 and 2038 instances. Female performances are also much more frequent than male ones, and gender often plays a role when training and evaluating models. For these reasons, several subsets of the dataset were composed by hand:
\begin{description}
\item[DampB] {Contains 20 full recordings per song (6000 in sum), both male and female.}
\item[DampBB] {Same as before, but phoneme instances were discarded until they were balanced and a maximum of 250,000 frames per phoneme where left, where possible. This data set is about 4\% the size of \textit{DampB}.}
\item[DampBB\_small] {Same as before, but phoneme instances were discarded until they were balanced and 60,000 frames per phoneme were left (a bit fewer than the amount contained in \textit{TIMIT}). This data set is about half the size of \textit{DampBB}.}
\item[DampFB and DampMB] {Using 20 full recordings per song and gender (6000 each), these data sets were then reduced in the same way as \textit{DampBB}. \textit{DampFB} is roughly the same size, \textit{DampMB} is a bit smaller because there are fewer male recordings.}
\item[DampTestF and DampTestM] {Contains one full recording per song and gender (300 each). These data sets were used for testing. There is no overlap with any of the training data sets.}
\end{description}
%Matt's data set??

\subsection{Structure of the final corpus}
%TODO: INTEGRATE!!!
\begin{description}

\item[Training data sets] In order to generate training data sets, we decided to balance the data by songs so that certain songs (and their phoneme distributions) would not be overrepresented. The least represented song in the original data set has 20 recordings. We therefore chose 20 to 23 recordings per song at random to generate a training data set, which we named \textbf{DampB}. For each song, as many different singers as possible were selected. Additionally, recordings with more ``Loves'' (user approval) were more likely to be selected (however, a large percentage of the original data set did not have such ratings). This resulted in a data set of 6,902 recordings.\\
 We then repeated this process, but only took recordings by singers of one gender into account. In this way, we created data sets of recordings by female and male singers only, which we named \textbf{DampF} and \textbf{DampM} respectively. Due to the gender split, there were fewer recordings of some of the songs available. Male singers in particular are underrepresented in the original data set. Therefore, \textbf{DampF} contains 6,564 recordings, while \textbf{DampM} contains 4,534. These sizes are roughly in the same range as for the mixed-gender data set, which enables the comparison of models trained on all three.
 \item[Test data sets] For testing new algorithms, we also created two small test data sets from the original \textit{DAMP} data set. These are, again, split by gender, and contain one recording per song, resulting in 301 recordings for the female data set and 300 for the male one (since there was one song with not enough available male recordings). We call them \textbf{DampTestF} and \textbf{DampTestM} respectively.\\
 For mixed-gender training, we suggest simply performing the testing on both test data sets.
\end{description}
A structured overview is given in table \ref{tab:corpus}.

\begin{table}
 \begin{center}
 \begin{tabular}{|c||c|c|}
  \hline
   \textbf{Gender} & \textbf{Training} & \textbf{Test} \\
  \hline
  \hline
  Both & DampB & \\
  & (6,902) & \\
  \hline
  Female & DampF & DampTestF \\
  & (6,654) & (301) \\
  \hline
  Male & DampM & DampTestM \\
  & (4,534) & (300) \\
  \hline
 \end{tabular}
\end{center}
\vspace{-10pt}
 \caption{Overview of the structure of the \textit{DAMP}-based phonetically annotated corpora, number of recordings in brackets.}
 \label{tab:corpus}
\vspace{10pt}
\end{table}



%\subsection{Aji's synthesized singing data set} \label{subsec:data_synth}
%Since it was not feasible to hand-annotate a large data set over the course of this work, another approach was the automatic generation of sung audio. The advantage of this approach is that the results can be assumed to be perfectly aligned to the given phonemes.\\
%For the generation of this data set, recordings from the previously described \textit{DAMP} data set were used. Their phonemes were automatically aligned, and an automatic transcription of the melody was performed. These two sources of data were then aligned to each other. This alignment did not need to be perfect, it just needed to produce a plausible combination of melody line and phonemes. For some songs in the dataset, these steps did not produce acceptable results and they had to be discarded, resulting in a corpus of 5163 recordings with phoneme and melody transcriptions.\\
%The result of this step was then fed into the \textit{Sinsy}\cite{sinsy}\footnote{\url{http://sinsy.jp/)}} singing synthesizer to generate new singing recordings. This synthesizer uses HMMs trained on MFCCs, $F_0$, and vibrato parameters for syllable-level synthesis, and provides one female singing voice. The resulting recordings are good in quality and relatively natural sounding, but appear to have a slight accent.\\
%The whole generation process of this data set was performed in collaboration with Adam Aji. The data set will be referred to as \textit{SYNTH}.

%\subsection{Choosing keywords}


%\section{``Real-world'' data sets}
\subsection{QMUL Expletive data set}
This data set consists of 80 popular full songs which were collected at Queen Mary University, most of them Hip Hop. 711 instances of 48 expletives were annotated on these songs. In addition, the matching textual, unaligned lyrics were retrieved from the internet. The audio is provided at $44,100 kHz$ sampling rate in stereo, but was also resampled to $16,000 Hz$ and a mono track for compatibility with some models.

%\subsection{``69 Love Songs" data set}
%``69 Love Songs'' is a 3-CD album by the band ``The Magnetic Fields'', which was released in 1999 and named one of the \textit{Rolling Stone}'s 500 Greatest Albums of All Time in 2012 \cite{rollingstone}. It contains 69 songs in various musical styles and instrumentations, performed by a variety of musicians, including five vocalists. The total duration is 2 hours and 52 minutes. The data set is interesting for the purposes of this work because the songs' lyrics all cover a similar theme - namely, love. A word count on the lyrics shows, for example, that the word ``love'' itself occurs \~225 times in these songs.\\
%Unaligned lyrics were retrieved from \url{http://stephinsongs.wiw.org}. A thorough semantic analysis can be found in \cite{book:beghtol}.

\chapter{Singing phoneme recognition} \label{chap:phonerec}
\section{Phoneme recognition using models trained on speech}
%timit
%hmm
%dnn 1024x850x1024
% tested: alignment acap
% phonerec acap, dampm/f
%todo: hmm vs dnn??
As a starting point, models for phoneme recognition were trained on speech data, specifically the \textit{TIMIT} corpus. As in many classical ASR approaches, HMMs were selected as a basis and trained using the Hidden Markov Toolkit (HTK)\cite{htk}.\\
Additionally, Deep Neural Networks (DNN) were trained for comparison with models trained on the other data sets. They had three hidden layers of 1024 nodes, 850 nodes, and 1024 nodes again.\\
In both cases, 13 MFCCs were extracted, and deltas and double-deltas were calculated, resulting in a feature vector of dimension 39. As the output, a set of 37 monophones based on the one used in CMU Sphinx\footnote{\url{http://cmusphinx.sourceforge.net/}} and in the \textit{TIMIT} annotations was used. The set is listed with examples in \ref{app:phones}.\\
%output ,mono tri
% results, align, phone
The HMMs were used for aligning text lyrics to audio in some of the consequent approaches. To verify the quality of such an alignment, this was tested on the part of the \textit{ACAP} singing corpus that has phoneme annotations. The result is shown as the blue bar of figure \ref{fig:res_alignment}: A mean alignment error of $0.16$ seconds. A small manual check suggests that this value may be in the range of the agreement of annotators. A closer inspection of the results also shows that the biggest contribution to this error occur because a few segments were heavily misaligned, whereas most of them are just slightly off.\\
%TODO: other figure
\begin{figure}
	\begin{center}
		\includegraphics[width=.4\textwidth]{images/res_alignment.png}
		\caption{Mean alignment error in seconds on the \textit{ACAP} data set. \textit{Timit} shows the result for the same models used for aligning the new \textit{DAMP}-based data sets.}
		\label{fig:res_alignment}
	\end{center}
\end{figure}
The DNN models were trained to perform phoneme recognition. The results for those trained on \textit{TIMIT} are shown as the leftmost bars in figures \ref{fig:res_phonerec_acap} and \ref{fig:res_phonerec}. The first figure displays the results on the \textit{ACAP} data set. The phoneme error rate is $1.06$, the weighted phoneme error rate is $.8$. In the second figure, the same evaluation is performed on the \textit{DAMP} test data sets, with similar results: A phoneme error rate of $1.3$, and a weighted phoneme error rate of $.93$. This demonstrates that the performance of models trained on speech leaves room for improvement when used for phoneme recognition in singing.\\
It can be assumed that models trained on better-matching conditions (i.e. singing) would perform much better at this task. The problem with this approach lies in the lack of data sets that can be used for these purposes. In contrast with speech, no large corpora of phonetically annotated singing are available. In the following sections, several workarounds for this problem are tested.


\begin{figure*}
	\centering
	\begin{subfigure}[t]{0.3\textwidth}
		\includegraphics[width=\textwidth]{images/res_phonerec_acap.png}
		\caption{Phoneme error rate}
		
	\end{subfigure}%
	\begin{subfigure}[t]{0.3\textwidth}
		\includegraphics[width=\textwidth]{images/res_phonerec_acap_w.png}
		\caption{Weighted phoneme error rate}
	\end{subfigure}
	\caption{Mean phoneme recognition results on the \textit{ACAP} data set using acoustic models trained on \textit{Timit} and the new \textit{DAMP}-based data sets.}\label{fig:res_phonerec_acap}
\end{figure*}

\begin{figure*}
	\centering
	\begin{subfigure}[t]{0.3\textwidth}
		\includegraphics[width=\textwidth]{images/res_phonerec.png}
		\caption{Phoneme error rate}
		
	\end{subfigure}%
	\begin{subfigure}[t]{0.3\textwidth}
		\includegraphics[width=\textwidth]{images/res_phonerec_w.png}
		\caption{Weighted phoneme error rate}
	\end{subfigure}
	\caption{Mean phoneme recognition results on the \textbf{DampTest} data sets using acoustic models trained on \textit{Timit} and the new \textit{DAMP}-based data sets.}\label{fig:res_phonerec}
\end{figure*}

\section{Phoneme recognition using models trained on ``songified" speech}
% time stretch, pitch shift
% variants
% dnn vs dbn??
% phonerec tested on: timit, acap
% todo: test on damp

When there is a scarcity of suitable training data, attempts are often made to generate such data artificially. For example, this is often done when models for noisy speech are required \cite{ntimit}\cite{aurora}. Inspired by this, one idea was making existing speech data sets more ``song-like'' and use these modified datasets to train models for phoneme recognition in singing. The \textit{TIMIT} corpus was once again used as a basis for this.\\

An overview of the approach is shown in figure \ref{fig:process_songify}. Five variants of the training part  \textit{TIMIT} were generated first. MFCC features were then extracted from these new datasets and used to train models.\\
Similarly, MFCCs are extracted from the \textit{TIMIT} Test set and from the \textit{ACAP} data set. The previously trained models are used to recognize phonemes on these test datasets. Viterbi decoding can then be used to generate phoneme sequences. Finally, the results are evaluated.
\setlength{\belowcaptionskip}{-0.4cm}
\begin{figure*}
 \begin{center}
                \includegraphics[width=0.7\textwidth]{images/process_songify.png}
                \caption{Overview of our phoneme recognition system}
                \label{fig:process_songify}
                 \end{center}
 \end{figure*}
 \setlength{\belowcaptionskip}{-0.1cm}
In order to make the training data more ``song-like'', several variants of this dataset were developed. Table \ref{tab:timit_variants} shows an overview over the five datasets generated from \textit{TIMIT} using three modifications. Dataset $N$ is the original \textit{TIMIT} training set. For dataset $P$, four of the eight blocks of \textit{TIMIT} were pitch-shifted. For dataset $T$, five blocks were time-stretched and vibrato was applied to two of them. In dataset $TP$, the same is done, except with additional pitch-shifting. Finally, dataset $M$ contains a mix of these modified blocks.\\
In detail, the modifications were performed in the following way:

\begin{description}
 \item[Time stretching] For time stretching, the phase vocoder from \cite{ellis_pvoc}, which is an implementation of the Flanagan/Dolson phase vocoder \cite{flanagan}\cite{dolson}, was used. This algorithm works by first performing a Short-Time Fourier Transform (STFT) on the signal and then resampling the frames to a different duration and performing the inverse Fourier transform.\\
 As demonstrated in section \ref{sec:speech_to_singing}, time variations in singing are mainly performed on vowels and are often much longer than in speech. Therefore, the \textit{TIMIT} annotations were used to only pick out the vowel segments from the utterances. They were modified randomly to a duration between $5$ and $100$ times the original duration and then re-inserted into the utterance. This effectively leads to more vowel frames in the training data, but since there is already a large amount of instances for each phoneme in the original training data, the effects of this imbalance should be negligible. 
 \item[Pitch shifting] To pitch-shift the signal, code from the freely available Matlab tool \textit{AutoTune Toy}\cite{autotunetoy}, which also implements a phase vocoder, was used. In this case, the fundamental frequency is first detected automatically. The signal is then stretched or expanded to obtain the new pitch and interpolated to retain the original duration.\\
 Using the \textit{TIMIT} annotations, utterances were split up into individual words, and then a pitch-shifted version of each word was generated and the results were concatenated. Pitches are randomly selected from a range between $60\%$ and $120\%$ of the original pitch.
 \item[Vibrato] The code for vibrato generation was also taken from \textit{AutoTune Toy}. It functions by generating a sine curve and using this as the trajectory for the pitch shifting algorithm mentioned above. A sine of amplitude $0.2$ and frequency $6 Hz$ was used.\\
 In singing, vibrato is commonly done on long sounds, which are usually vowels. Since spoken vowels are usually very short, vibrato cannot be perceived on them very well. Therefore, vibrato was only added when time stretching was also applied. Vibrato was then added to the extracted and stretched vowels.
\end{description}



\begin{table}
 \begin{center}
  \begin{tabular}{|c||c|c|c|c|c|}
  \hline
   & \textbf{N} & \textbf{P} &\textbf{T} &\textbf{TP} &\textbf{M} \\
  \hline
  DR1 & N & N & N & N & N  \\
  DR2 & N & N & N & N & N \\
  DR3 & N & N & N & N & P\\
  DR4 & N & N & T & TP & TV \\
  DR5 & N & P & T & TP & TPV \\
  DR6 & N & P & T & TP & TV \\
  DR7 & N & P & TV & TPV & P \\
  DR8 & N & P & TV & TPV & TPV \\
  \hline
 \end{tabular}
\end{center}
 \caption{The five TIMIT variants that were used for training (rows are TIMIT blocks, columns are the five datasets).
  Symbols: N - Unmodified; P - Pitch-shifted; T - Time-stretched; V - Vibrato}
 \label{tab:timit_variants}
\end{table}



\section{Phoneme recognition on synthesized singing}
% sinsy; female only
% dnn
%phonerec tested on:
% matched set, acap, dampf
%problem: correct instead of w_per!!

\section{Phoneme recognition using models trained on a-capella singing} \label{sec:phonerec_acap}
%how created?
% dnn
%phonerec tested on:
% acap, dampf/m
%alignment tested on acap; also: hmms!!

\subsection{Corpus construction}

%pre-existing; copy old paragraphs here
As a basis for our phonetically annotated data set, we used the \textit{DAMP} data set, which is freely available from Stanford University\footnote{\url{https://ccrma.stanford.edu/damp/}}\cite{phdthesis:jeffreysmith}. This data set contains more than 34,000 recordings of amateur singing of full songs (3 to 5 minutes duration) with no background music, which were obtained from the \textit{Smule Sing!} karaoke app. Each performance is labeled with metadata such as the gender of the singer, the region of origin, the song title, etc. The singers performed 301 different English language pop songs. The recordings have good sound quality with little background noise, but come from a lot of different recording conditions.


\begin{figure*}
 \begin{center}
                \includegraphics[width=.8\textwidth]{images/overview_bootstrap.png}
                \caption{An overview of the alignment process. The right-hand part represents the optional bootstrapping.}
                \label{fig:process}
                 \end{center}
 \end{figure*}
No lyrics annotations are available for this data set, but we obtained the textual lyrics from the \textit{Smule Sing!} website\footnote{\url{http://www.smule.com/songs}}.  All of them were English-language songs. These lyrics were mapped to their phonetic content using the CMU Pronouncing Dictionary\footnote{\url{http://www.speech.cs.cmu.edu/cgi-bin/cmudict}} with some manual additions of unusual words. This dictionary has a phoneme set of 39 phonemes.\\
We then performed an automatic alignment of these lyrics to the \textit{DAMP} audio.\\
As a basis, a monophonic HMM acoustic model trained on the \textit{Timit} speech corpus was used \cite{timit}. The model training and the alignment were done using the HTK framework \cite{htk}, which uses Viterbi alignment as its algorithm. MFCCs and their deltas and double-deltas were used as features. Alignment was performed on the word and phoneme levels. This is the same principle of so-called ``Forced Alignment" that is commonly used in Automatic Speech Recognition \cite{book:jurafsky} (although it is usually done on shorter utterances). \\
On top of this, we carried out several different alignment strategies:
\begin{description}
\item[Monophones vs. Triphones] We tested aligning both monophones (i.e. one state per phoneme) and triphones (i.e. three states per triphone modeling the start, middle, and end phases). Since \textit{Timit} only contains monophone annotations, this was done by first splitting the phoneme time frames evenly through three, and then re-training the \textit{Timit} acoustic model and re-aligning the data set (with the assumption that the transitions between the triphone states would be ``pulled'' to the correct times).  
\item[One-pass alignment vs. Bootstrapping] On top of a one-pass alignment using the Viterbi algorithm, we also investigated whether the acoustic models could be bootstrapped to improve the alignment. To clarify: We first performed the alignment on the \textbf{Damp} data sets using the \textit{Timit} models described above, then trained acoustic models on the resulting phoneme annotations. Then, those models were used to re-align the \textbf{Damp} data, which was again used to train another model. This was done over three iterations.\\
A modified version of the alignment algorithm was used for doing alignment with the models trained on the \textbf{Damp} data sets. This approach is based on doing Dynamic Time Warping on the generated phoneme posteriorgrams without punishing very long states. This is similar to the approach described in \cite{kruspe_lyrics_retrieval}.
\end{description}
A graphical overview of the alignment process is given in figure \ref{fig:process}.\\
Of course, errors cannot be avoided when doing automatic forced alignment. All in all, there were now four combinations of these strategies, which we compared. In section \ref{sec:validation}, we describe how this alignment procedure was validated and what strategies performed best.\\
Since there is usually a large number of recordings of the same song, we considered using this information to improve the alignment results, e.g. by averaging timestamps over the alignments of several recordings. We did not do this in this work because recordings tend to have different offsets from the beginning (i.e. silence in the beginning), and the singers also do not necessarily pronounce phonemes at the same time. This might be an avenue for future research, though.
\section{Validation}\label{sec:validation}
Validating the phoneme annotations created in this way is not trivial since there is no ground truth to base them on. We therefore used a two-pronged approach: We first tested the same alignment algorithm on a different, small, manually annotated data set. Second, we trained new acoustic models on the newly generated training data sets. We then performed phoneme recognition on the test data sets, and compared the results to the expected phoneme strings (which are known since we have the matching lyrics).
\subsection{Validation data}\label{subsec:validation_data}
For testing the alignment approach, we used a small data set of the vocal tracks of 15 pop songs, which were hand-annotated with phonemes and words. This data set was first presented in \cite{jens}. We call it \textit{ACAP}.
%Despite the small size, we provide results on this data set for comparison with our previous approaches, and because the ground truth annotations can be assumed to be correct (in contrast with the automatically generated annotations of the \textit{Damp}-based data sets).
%problem: no valid ground truth
\subsection{Alignment validation}
We first tested the same alignment approach that was used to create the new \textit{DAMP}-based data sets on the \textit{ACAP} data set. To recap: This approach employs models trained on the \textit{Timit} speech corpus, which are used for Viterbi alignment of the known phonemes to the singing. The result of this is then compared to the manual annotations by calculating the difference between each expected and predicted phoneme transition. 



We then tested various models trained on the new \textbf{DampB}, \textbf{DampF}, and \textbf{DampM} training data sets for the same task. The results are also shown in figure \ref{fig:res_alignment}.\\
Models trained on the monophonic alignments of \textbf{DampB} and \textbf{DampF/M} perform slightly better at this task with mean errors of $0.15$ and $0.13$ seconds respectively (for the gender-specific models, those values were only calculated for the songs of the matching gender in \textit{ACAP}). The triphone version of \textbf{DampB} performs even better, with a mean error of $0.1$. We believe this might be because dedicatedly training the model for the start and end parts of phonemes might make the alignment approach more accurate at finding start and end points.\\
Finally, we also tested the model trained on \textbf{DampB} over three iterations. This model performs much worse at this task. This might happen because errors in the original alignment of the phonemes may become amplified over these iterations.

\subsection{Validation of phoneme recognition}
To obtain a clearer picture of the quality of the new data sets, we also performed phoneme recognition experiments on both \textit{ACAP} and the \textbf{DampTest} corpora. This was possible even though there are no manual annotations for the \textbf{DampTest} sets because the expected phonemes are available from the textual lyrics. The phoneme error rate and the weighted phoneme error rate were used as evaluation measures (see \cite{kruspe_phonerec}).\\
The results for \textit{ACAP} are shown in figure \ref{fig:res_phonerec_acap}. In general, models trained on \textbf{DampB} performed much better at phoneme recognition than those trained on \textit{Timit}. Compared to these speech-based models, the phoneme error rate falls from $1.06$ to $0.77$, while the weighted phoneme error rate falls from $0.8$ to $0.59$. As can be seen from both evaluation measures, using triphone alignments instead of monophones does not improve the results in this case. This contrasts with the better alignment results. We think this might happen because more classes cause more confusion in the model, even though the triphone results were downmapped to monophones for calculating the evaluation measures.\\
As already seen in the alignment validation results, training models on \textbf{DampB} over three iterations actually degrades the result (``DampB (3)''). Again, we suspect that this happens because phoneme alignment errors become amplified in this way. Interestingly, this does not seem to happen for the triphone models, perhaps because the three classes per phoneme help to alleviate each other's errors.\\
The results for the same procedure on the \textbf{DampTest} sets are shown in figure \ref{fig:res_phonerec}. Results over \textbf{DampTestF} and \textbf{DampTestM} were averaged. This figure additionally shows the results for models trained on the gender-specific \textbf{DampF} and \textbf{DampM} data sets. These models were only tested on \textbf{DampTestF} and \textbf{DampTestM} respectively, and then the results were averaged.\\
We can observe the same general trend for these results: The phoneme error rate falls from $1.3$ to $0.87$ when compared to models trained on \textit{Timit}, with the weighted phoneme error rate decreasing from $0.93$ to $0.65$. Using triphones does not contribute to the result, and neither does the three-iteration bootstrapping process for training acoustic models. Interestingly, not even the gender-specific models improve the result. As already described in \cite{kruspe_phonerec}, this effect might occur because the range of pitch and expressions is much wider in singing than in speech, and therefore gender-specific models may not actually learn as much added helpful information.



\subsection{Error sources}
Of course, with an automatic alignment algorithm like this, errors cannot be avoided. To acquire a clearer picture of the reasons for the various misalignments, we had a closer look at the audio data where they occurred. Some sources of error repeatedly stuck out:
%\begin{compactdesc}
\begin{description}
 \item[Unclear enunciation]{Some singers pronounced words very unclearly, often focusing more on musical performance than on the lyrics.}
 \item[Accents]{Some singers sung with an accent, either their natural one or imitating the one used by the original singer of the song.}
 \item[Young children's voices]{Some recordings were performed by young children.}
 \item[Background music]{Some singers had the original song with the original singing running in the background.}
 \item[Speaking in breaks]{Some singers spoke in the musical breaks.}
 \item[Problems in audio quality]{Some recordings had qualitative problems, especially loudness clipping.}
%\end{compactdesc}
\end{description}
For most of these issues, more robust phoneme recognizers would be helpful. For others, the algorithm could be adapted to be robust to extraneous recognized phonemes (particularly for the speaking problem). If possible, a thorough manual check of the data would be very helpful as well.



\section{Conclusion}

%TODO: Erw�hnen, welche balancierten Subsets genutzt wurden

\chapter{Sung Language Identification} \label{chap:langid}
%Sung language identification is the task of automatically determining the language of a sung recording. For this task, two algorithms were developed: One based on i-vector extraction, and one based on analysis of the statistics of phoneme posteriorgrams. They will be described in detail in this chapter.\\
%In both cases, the \textit{NIST2003LRE} and \textit{OGIMultilang} corpora were used for testing the algorithms on speech, and the \textit{YTAcap} data set was used for singing (see chapter \ref{chap:datasets}).
\begin{figure}
	\begin{center}
		\includegraphics[width=1\textwidth]{images/process_training_lid.png}
		\caption{Schematic of the training procedure for language identification.}
		\label{fig:process_training_lid}
	\end{center}
\end{figure}
\begin{figure}
	\begin{center}
		\includegraphics[width=1\textwidth]{images/process_classification_lid.png}
		\caption{Schematic of the classification procedure for language identification.}
		\label{fig:process_classification_lid}
	\end{center}
\end{figure}

Language identification has been extensively researched in the field of Automatic Speech Recognition since the 1980's. A number of successful algorithms has been developed over the years. An overview over the fundamental techniques is given by Zissman in \cite{zissman}.\\
Fundamentally, four properties of languages can be used to discriminate between them:
\begin{description}
	\item[Phonetics] The unique sounds that are used in a given language.
	\item[Phonotactics] The probabilities of certain phonemes and phoneme sequences.
	\item[Prosody] The ``melody'' of the spoken language.
	\item[Vocabulary] The possible words made up by the phonemes and the probabilities of certain combinations of words.
\end{description}
Even modern systems mostly focus on phonetics and phonotactics as the distinguishing factors between languages. Vocabulary is sometimes exploited in the shape of language models.\\
In ASR, the standard technique for language identification is Parallel Phone Recognition followed by Language Modeling (PPRLM). In this approach, acoustic and language models are trained for each language (or, in some cases, only the language models are different and just one acoustic model is used) . Unseen examples are then run through each model or combinations thereof, and the result with the highest likelihood determines the language (e.g. \cite{lid_li_ma} and \cite{lid_matejka}).\\
Other approaches directly train models for each language on the feature vectors (e.g. GMMs). This technique can be considered a ``bag of frames'' approach, i.e. the single data frames are considered to be statistically independent of each other. The trained models then describe probability densities for certain acoustic characteristics of each language. GMM approaches used to perform worse than their PPRLM counterparts, but the development of new features has made the difference negligible \cite{singer}. They are, in general, easier to implement since only audio examples and their language annotations are required. Allen et al. \cite{allen} report results of up to $76.4\%$ accuracy for ten languages. Different backend classifiers, such as Multi-Layer Perceptrons (MLPs) and Support Vector Machines (SVMs) \cite{campbell}, have also been used successfully instead of GMMs.\\
%HERE??
In this work, an approach that trains directly on the acoustic characteristics using i-vector extraction and SVMs is presented. The modifications to the general processing chain are presented in figures \ref{fig:process_training_lid} and \ref{fig:process_classification_lid}.\\
 
Additionally, a second approach based on phoneme posteriorgrams is tested. Statistics from the posteriorgrams are calculated, and then a second model is trained on these. Similar methods have been developed in ASR: Berkling presented an approach that uses sequences of recognized phonemes to discriminate between two languages (English and German), either with statistical modeling or with Neural Networks \cite{phdthesis:berkling_phd}. Mean errors of $0.12$ and $0.07$ on unseen data are achieved for the statistical approach and the Neural Network approach respectively when enough training data is available.\\
Li, Ma, and Lee present a system where acoustic inputs are tokenized into acoustic words, which do not necessarily correspond to phonetic n-grams. Then, language classifiers are trained on statistics of the acoustic words \cite{lid_li_ma_lee}. They obtain an equal error rate of $0.05$ for six languages using a universal phoneme recognizer for tokenization and SVMs for backend language recognition. Peche et al. \cite{peche} attempt a similar approach on languages with limited resources. The performance remains good even when only acoustic models trained on different languages are used.\\
In all of these approaches, tokenization of some sort is performed using the acoustic models. Since phoneme recognition on singing is still relatively unreliable, statistics are calculated directly on the phoneme posteriors in this work.\\


For evaluation, all test examples are classified into exactly one language class. Then, the accuracy (i.e. the average retrieval) is calculated:
 \begin{equation}
    Accuracy = \frac{TP}{N}
 \end{equation}  
 where $TP$ are the True Positives, and $N$ is the number of all documents. In ASR, the average cost measure as recommended in \cite{nist_cavg} is also used widely now; however, to remain in line with other sung language identification approaches such as those described in chapter \ref{chap:sota}, the accuracy was still used in this work.\\
 
 In both approaches, the \textit{NIST2003LRE} and \textit{OGIMultilang} corpora were used for testing the algorithms on speech, and the \textit{YTAcap} data set was used for singing (see chapter \ref{chap:datasets}). All results are obtained using 5-fold cross-validation - i.e., models are trained on 4/5 of each data set, then the remaining 1/5 is classified with the model. This is done 5 times until each utterance has been classified. This was necessary because the data sets are relatively small, and separating them into training and test sets would not have provided enough results for a meaningful evaluation.




\section{Sung language identification using i-vectors}
%known speakers - unknown speakers - whole documents


As described in section \ref{subsec:tech_ivector}, i-vector extraction is a feature dimension reduction technique that was originally developed for speaker recognition, but has since then been employed successfully for other tasks, including language identification. After the publication of this approach for i-vector extraction for sung language identification, it also started being used for other MIR tasks, such as artist recognition and similarity calculation \cite{eghbal-zadeh1}\cite{eghbal-zadeh2}.

\subsection{Proposed system}
\label{sec:system}
Figure \ref{fig:lid_ivectors} shows a rough overview over the i-vector classification system. 
%\begin{figure}[ht]
%\centerline{\includegraphics[scale=0.3]%{overview.png}}
%\caption{\label{fig:overview}{\it Overview of the steps of our classification system.}}
%\end{figure}
\begin{figure*}
	\begin{center}
		\includegraphics[width=.7\textwidth]{images/lid_ivectors.png}
		\caption{Overview of the process for language identification using i-vector extraction.}
		\label{fig:lid_ivectors}
	\end{center}
\end{figure*}

A number of features were extracted from each audio file. Table \ref{tab:ivec_configs} shows an overview over the various configurations used in training.
\begin{table}[h!tbp]
\scriptsize
  \begin{center}
     \begin{tabular}{|c|c|c|}\hline
     Name & Description & Dimensions \\ \hline
      MFCC  & MFCC, 20 coefficients & 20 \\
      MFCCDELTA  & MFCC, 20 coefficients, deltas and double-deltas & 60 \\
      MFCCDELTASDC  & MFCCDELTA+SDC & 117 \\
      SDC  & SDC with configuration $7-1-3-7$ & 91 \\
      RASTA-PLP & PLP with RASTA processing, model order 13 with deltas and double-deltas & 39 \\
      RASTA-PLP36  & PLP with RASTA processing, model order 36 with deltas and double-deltas  & 96 \\
      PLP  & PLP without RASTA processing, model order 13 deltas and double-deltas  & 39 \\
      COMB & PLP+MFCCDELTA & 99 \\ \hline
    \end{tabular}
  \end{center}
    \caption{{Feature configurations used in language identification training.}}
  \label{tab:ivec_configs}
\end{table}

For classification, Multi-Layer Perceptrons (MLPs) and Support Vector Machines (SVMs) are tested. The MLPs are fixed at three layers, with the middle layer having a dimension of 256. Additional layers do not improve the result. A larger middle layer only improves it slightly. The SVM parameters are determined using a grid-search. For each of those classifiers and data sets, all feature combinations listed in table \ref{tab:ivec_configs} are tested directly and with i-vector processing.\\


\subsection{Experiments with known speakers}
In the first experiment, MLP and SVM models are trained on randomly selected folds of the training data sets. This means that recordings by the same speaker are spread out between the training and test data sets. In theory, i-vectors could be particularly susceptible to capturing speaker characteristics instead of language characteristics, leading to evaluation results that are not representative of results for unknown speakers. However, this effect is often ignored in literature \cite{Martin2010}, and an argument can be made that partially training models on speaker properties is realistic for some use cases (i.e. when many of the expected speakers for each language are already known).\\

\begin{figure}[h]
       \centering
      \begin{subfigure}[c]{0.3\textwidth}
                \includegraphics[width=\textwidth]{images/lid_exp1_nist.png}
                \caption{NIST2003LRE}
                \label{fig:lid_exp1_nist}
        \end{subfigure}%
        \begin{subfigure}[c]{0.3\textwidth}
                \includegraphics[width=\textwidth]{images/lid_exp1_ogi.png}
                \caption{OGIMultilang}
                \label{fig:exp1_ogi}
        \end{subfigure}
                \begin{subfigure}[c]{0.3\textwidth}
                \includegraphics[width=\textwidth]{images/lid_exp1_yt.png}
                \caption{YTAcap}
                \label{fig:exp1_yt}
        \end{subfigure}
        \caption{Results using MLP models on all three language identification data sets, with or without i-vector processing.}\label{fig:lid_exp1}
\end{figure}

The results for the MLP training on the \textit{NIST2003LRE}, \textit{OGIMultilang}, and \textit{YTAcap} data sets are shown in figure \ref{fig:lid_exp1} in terms of accuracy (average retrieval when all documents are classified into exactly one language). \\
As shown in figure \ref{fig:lid_exp1_nist}, the MLP does not produce good results on the \textit{NIST2003LRE} database for any of the feature combinations. \textit{NIST2003LRE} is the smallest of the data sets by a large margin. Since a relatively high-dimensional model is used, this is probably a case of overtraining. The i-vector processing step reduces the training data even further, thus aggravating the problem.\\
The \textit{OGIMultilang} data set contains roughly 4 times as much data as the \textit{NIST2003LRE} set. With enough data, training an MLP classifier works a lot better. Without i-vector processing, this approach still only reaches about 52\% accuracy. i-Vector extraction improves the system massively. The best feature configurations are RASTA-PLP (82\%), PLP (80\%), and COMB (80\%).\\
As with all other experiments, the task becomes harder when attempted on singing data. The results on the \textit{YTAcap} data set are worse than those on \textit{OGIMultilang}, even though they contain a similar amount of data. The best result without i-vector extraction is still obtained using the COMB feature configuration at 56\% accuracy. Similar to the \textit{OGIMultilang} experiment, i-vector extraction yields a large improvement. COMB remains the best configuration, now at 77\% accuracy.\\

\begin{figure}[h]
       \centering
      \begin{subfigure}[c]{0.3\textwidth}
                \includegraphics[width=\textwidth]{images/lid_exp1a_nist.png}
                \caption{NIST2003LRE}
                \label{fig:lid_exp1a_nist}
        \end{subfigure}%
        \begin{subfigure}[c]{0.3\textwidth}
                \includegraphics[width=\textwidth]{images/lid_exp1a_ogi.png}
                \caption{OGIMultilang}
                \label{fig:exp1a_ogi}
        \end{subfigure}
                \begin{subfigure}[c]{0.3\textwidth}
                \includegraphics[width=\textwidth]{images/lid_exp1a_yt.png}
                \caption{YTAcap}
                \label{fig:exp1a_yt}
        \end{subfigure}
        \caption{Results using SVM models on all three language identification data sets, with or without i-vector processing, with speakers shared between training and test sets.}\label{fig:lid_exp1a}
\end{figure}


Figure \ref{fig:lid_exp1a} shows the results for the SVM models trained on the same data sets. In general, these models are able to capture the language boundaries better.\\
In contrast to the MLP experiment, SVMs produce good results on the \textit{NIST2003LRE} data set for all of the features. They are able to discriminate very well on this small, clean data set. The best result with i-vector processing is $86\%$ accuracy for MFCC features. When using i-vectors, a 93\% accuracy is achieved with PLP features. This may, in fact, be close to the upper bound for the classification here; further analysis shows that misclassified recordings often mainly consist of laughter or very few words.\\
The \textit{OGIMultilang} corpus is bigger and more varied than the \textit{NIST2003LRE} corpus, which makes it harder to classify. As shown, the high-dimensional pure features do not perform as well, with a maximum accuracy of 68\% for MFCCs and RASTA-PLPs with 36 coefficients. Using i-vector extraction improves the result by a large margin. Feature-wise, PLPs without RASTA processing work best at a result of 82\% accuracy. 
%CHECK!!!
MFCC and SDC features did not work quite as well, but did not hurt the result either when combined with PLPs (COMB result). It is interesting to see that the i-vector extraction decreased the results for MFCCs, the feature that worked best without it.\\
%CHECK!!
Similar to the \textit{OGIMultilang} corpus, the \textit{YTAcap} corpus provides very complex and varied data. The same effects occur with the direct feature training here, too: RASTA-PLPs with 36 coefficients provide the best results, but the accuracy is not very high at 68\%. i-Vector extraction once again serves to improve the result. The highest results when using i-vector extraction is a 75\% accuracy when using PLP without RASTA processing, or 77\% for the COMB configuration.

\subsection{Experiments with unknown speakers}

In order to find out what influence the speaker characteristics had on the result, the same experiments were then repeated with training and evaluation sets that strictly separated speakers. This experiment was not performed for the \textit{NIST2003LRE}�corpus because no speaker information is available for it. Since the SVM models performed better in the previous experiment, only these models were tested.\\

\begin{figure}[h]
       \centering
        \begin{subfigure}[c]{0.4\textwidth}
                \includegraphics[width=\textwidth]{images/lid_exp2_ogi.png}
                \caption{OGIMultilang}
                \label{fig:exp2_ogi}
        \end{subfigure}
                \begin{subfigure}[c]{0.4\textwidth}
                \includegraphics[width=\textwidth]{images/lid_exp2_yt.png}
                \caption{YTAcap}
                \label{fig:exp2_yt}
        \end{subfigure}
        \caption{Results using SVM models, with or without i-vector processing, with speakers separated between training and test sets.}\label{fig:lid_exp2}
\end{figure}

The results are shown in figure \ref{fig:lid_exp2}. In general, all configurations perform worse, indicating that some of the characteristics learned by the models come from the speakers rather than the languages. Apart from this, the general trends for the features remain the same, and i-vector extraction still improves the over-all results.\\

On the \textit{OGIMultilang} corpus, the best result is still obtained with RASTA-PLP features and i-vector processing, but the accuracy falls by around 8 percent points to $75\%$. On \textit{YTAcap}, the effect is even worse: From an accuracy of $77\%$ with the mixed condition, the result decreases to $61\%$ for the separated condition. The reason for this is probably the wider signal variety in singing as opposed to speech; additionally, \textit{YTAcap} also possesses a wider range of recording conditions than the controlled telephone conditions of \textit{OGIMultilang}. Arguably, the solution for this effect would be the use of larger training data sets, which would be able to cover these acoustic and performance conditions better. Conversely, as the previous results show, the approach produces better results when an application scenario can be limited to a range of known speakers, or at least recording conditions (as in the \textit{NIST2003LRE} experiment).


%TODO: info about speakers and material per speaker
\subsection{Experiments with utterances combined by speakers}

All previous experiments were performed on relatively short utterances of a few seconds in duration. In many application scenarios, much more audio data is available to make a decision about the language. In particular, songs are usually a few minutes in length, and in many cases, only one result per document (= song) is required.\\
For this reason, results for the \textit{YTAcap} data set are taken from the previous experiment and a majority voting decision is made for each song (and therefore also for each singer). For the \textit{OGIMultilang} corpus, results for all utterances by the same speaker are aggregated in the same fashion, resulting in similar durations of audio. (Again, this experiment was not performed with the \textit{NIST2003LRE} corpus due to the lack of speaker information).\\

\begin{figure}[h]
       \centering
           \begin{subfigure}[c]{0.4\textwidth}
                \includegraphics[width=\textwidth]{images/lid_exp3_ogi.png}
                \caption{OGIMultilang}
                \label{fig:exp3_ogi}
        \end{subfigure}
                \begin{subfigure}[c]{0.4\textwidth}
                \includegraphics[width=\textwidth]{images/lid_exp3_yt.png}
                \caption{YTAcap}
                \label{fig:exp3_yt}
        \end{subfigure}
        \caption{Document-wise results using SVM models, with or without i-vector processing.}\label{fig:lid_exp3}
\end{figure}

The results are shown in figure \ref{fig:lid_exp3}. Overall, aggregation of multiple utterances by the same speakers balances out some of the speaker-specific effects seen in the previous experiment. Taking more acoustic information into account, the models are able to determine the language with higher accuracy.\\
On the \textit{OGIMultilang} corpus, the result is even better than on the condition with known speakers. The best result rises from $75\%$ accuracy for short utterances to $92\%$ for the aggregated documents (both with the RASTA-PLP feature). On the \textit{YTAcap} data sets, the aggregated result is $69\%$ (compared to $60\%$ for line segments).\\

As suggested in the previous section, the approach produces results that are usable in practice when the problem can be narrowed down, e.g. to known speakers or recording conditions. As this experiment shows, useful results can also be obtained when longer sequences are available for analysis. 


\

%nicht n�tig f�r nist - kleine datenmengen gehen so mit svm. bei mlp aber overfitting.
%ogi: ivec erh�ht Ergebnis massiv. besser als irmfsp (warum?)
%yt: �hnlich ogi. Ergebnisse > sota. 
%features: plp gut, am besten ohne Rasta (warum?). mehr coeffs bringen aber nichts. mfcc f�r sich auch gut, mit delta noch besser, teilweise auch mit sdc.
%combi aus plp_norasta und mfccdelta funktioniert am besten - 2 verschiedene Aspekte abgedeckt (warum?)
%insgesamt: schnelleres training und weniger Speicherplatz als volle features, Irrelevanz reduziert. �hnliche Ergebnisse wie pprlm, aber viel einfacher. weniger Annotationen und leichter zu implementieren.




\section{Sung language identification using phoneme recognition posteriors}\label{sec:lid_stats}
Another developed approach is based upon phoneme statistics derived from phoneme posteriorgrams. To obtain representative statistics for model training, relatively long observations are necessary, but, as described in the previous section, this is the case for many applications, for example when considering song material (e.g. songs of 3-4 minutes in duration). On the other hand, phoneme posteriorgrams need to be calculated for a number of other tasks, such as keyword spotting or lyrics-to-audio alignment. \\

\begin{figure*}
	\begin{center}
		\includegraphics[width=1\textwidth]{images/process_lid_stats.png}
		\caption{Overview of the process for language identification using phoneme statistics.}
		\label{fig:lid_statistics_process}
	\end{center}
\end{figure*}

An overview of the approach is shown in figure \ref{fig:lid_statistics_process}. Posteriorgrams are generated on the test data sets \textit{YTAcap} and \textit{OGIMultilang} using the acoustic models trained on the \textit{TIMIT} speech data set and on the \textit{DAMP} singing data set as described in section \ref{sec:phonerec_acap}. To facilitate the following language identification, phoneme statistics are then calculated in two different ways:
\begin{description}
  \item[Document-wise statistics]{Mean and variances of the phoneme likelihoods over whole songs or sets of utterances of a single speaker are calculated. This results in just two feature vectors per document (one for the means, one for the variances).}
  \item[Utterance-wise statistics]{Means and variances of the phoneme likelihoods over each utterance are calculated (or, in the case of \textit{YTAcap}, over each song segment). For further training, the resulting vectors for each speaker/song (= document) are used as a combined feature matrix. As a result, no overlap of speakers/songs is possible between the training and test sets.}
\end{description}
Naturally, relatively long recordings are necessary to produce salient statistics. For this reason, the aggregation by speaker/song is done in both cases rather than treating each utterance separately. \\

Then, Support Vector Machine (SVM) models are trained on the calculated statistics in both variants with the three languages as annotations. Unknown song/speaker documents can then be subjected to the whole process and classified by language.\\
%Again, all results are obtained using 5-fold cross-validation - i.e., SVMs are trained on 4/5 of each corpus, then the remaining 1/5 is classified with the model. This is done 5 times until each song/speaker document has been classified.

\subsection{Language identification using document-wise phoneme statistics}
In the first experiment, SVM classifiers are trained on the document-wise phoneme statistics, and classification is also performed on a document-wise basis (i.e., only one mean and one variance vector per document). The results are shown in figure \ref{fig:lid_statistics_res1}.\\
On the singing test set, results are worst when using acoustic models trained on \textit{TIMIT} at just $53\%$ accuracy, and become better when using the model trained on the ``songified'' \textit{TIMIT} variant \textit{TimitM} (see section \ref{sec:phonerec_songify}), or on the small selection of the singing training set \textit{DampBB\_small} at an accuracy of $59\%$ each. The best result of $63\%$ accuracy is achieved when the models are trained on the full singing data set.\\
Surprisingly, the results on the \textit{OGIMultilang} corpus also improve from $75\%$ with the \textit{TIMIT} models to $84\%$ using the \textit{DampB} models. Since \textit{TIMIT} is a very ``clean'' data set, training on the singing corpus provides some more phonetic variety, acting as a sort of data augmentation. This is especially important in this context where phonemes are recognized in three different languages.\\
On both corpora, there is no noticeable bias of the confusion matrix - i.e., the confusions are spread out evenly. This is particularly interesting when considering that the acoustic models were trained on English speech or singing only.
%\begin{figure}[h]
 %       \centering
  %      \begin{subfigure}[c]{0.25\textwidth}
  %              \includegraphics[width=\textwidth]%{figs/stat_acc.png}
   %             \caption{Accuracy}
   %             \label{fig:stat_acc}
  %      \end{subfigure}%
   %     ~ %add desired spacing between images, e. g. ~, \quad, \qquad, \hfill etc.
          %(or a blank line to force the subfigure onto a new line)
   %     \begin{subfigure}[c]{0.25\textwidth}
     %           \includegraphics[width=\textwidth]%{figs/stat_cavg.png}
   %             \caption{Average cost}
  %              \label{fig:stat_cavg}
 %       \end{subfigure}
 %       \caption{Results using document-wise phoneme statistics generated with various acoustic models.}\label{fig:res_stat}
%\end{figure}

\begin{figure*}
	\begin{center}
		\includegraphics[width=.4\textwidth]{images/lid_statistics_res1.png}
		\caption{Results using document-wise phoneme statistics generated with various acoustic models.}
		\label{fig:lid_statistics_res1}
	\end{center}
\end{figure*}

\subsection{Language identification using utterance-wise phoneme statistics}
Next, language identification was performed with models trained on the statistics of each utterance contained in the document. The recognition process is still performed on the whole document. The results are reported in figure \ref{fig:lid_statistics_res2}.\\
Phoneme statistics are not as representative when computed on shorter inputs, but they provide more information for the backend model training when utilized as a combined feature matrix for a longer document. The results on singing improve slightly (significantly) to $63\%$ accuracy with the acoustic model trained on the small singing corpus (\textit{DampBB\_small}) and decrease insignificantly for the \textit{DampB} model ($61\%$). However, on the speech corpus, the best result rises to $90\%$.
%why??
%\setlength{\belowcaptionskip}{-0.3cm}
%\begin{figure}[h]
 %       \centering
 %       \begin{subfigure}[c]{0.25\textwidth}
  %              \includegraphics[width=\textwidth]%{figs/stats_acc.png}
  %              \caption{Accuracy}
 %               \label{fig:stats_acc}
%        \end{subfigure}%
 %       ~ %add desired spacing between images, e. g. ~, \quad, \qquad, \hfill etc.
          %(or a blank line to force the subfigure onto a new line)
 %       \begin{subfigure}[c]{0.25\textwidth}
 %               \includegraphics[width=\textwidth]%{figs/stats_cavg.png}
 %               \caption{Average cost}
  %              \label{fig:stats_cavg}
  %      \end{subfigure}
  %      \caption{Results using utterance-wise phoneme statistics generated with various acoustic models.}\label{fig:res_stats}
%\end{figure}
%\setlength{\belowcaptionskip}{-0.1cm}
\begin{figure*}
	\begin{center}
		\includegraphics[width=.4\textwidth]{images/lid_statistics_res2.png}
		\caption{Results using utterance-wise phoneme statistics generated with various acoustic models.}
		\label{fig:lid_statistics_res2}
	\end{center}
\end{figure*}

\subsection{For comparison: Results for the i-vector approach}
For comparison, models from the previous approach were also trained on the same time scales. i-Vectors were calculated on the utterance- or the document-wise scale. This was done for PLP and MFCC features. The resulting i-vectors were then used to train SVMs in the same manner as in the previous experiments. (The difference here is that the models are already trained on the aggregated i-vectors, either with those for a whole document or with all i-vectors of the utterances constituting each document aggregated). The results are shown in figure \ref{fig:lid_statistics_res3}.\\
The best result obtained on \textit{YTAcap} data set is $68\%$ accuracy. This is only 5 percent points higher than the approach based on phoneme statistics, which is easier to implement. On the \textit{OGIMultilang} corpus, the difference is only 3 percent points ($93\%$). Of course, the advantage of the i-vector approach is that it can also be performed on much shorter inputs.\\

\begin{figure*}
	\begin{center}
		\includegraphics[width=.4\textwidth]{images/lid_statistics_res3.png}
		\caption{Results using utterance- and document-wise i-vectors calculated on PLP and MFCC features.}
		\label{fig:lid_statistics_res3}
	\end{center}
\end{figure*}
%\begin{figure}[h]
 %       \centering
%        \begin{subfigure}[c]{0.25\textwidth}
 %               \includegraphics[width=\textwidth]%{figs/ivec_acc.png}
  %              \caption{Accuracy}
   %             \label{fig:ivec_acc}
  %      \end{subfigure}%
  %      ~ %add desired spacing between images, e. g. ~, \quad, \qquad, \hfill etc.
          %(or a blank line to force the subfigure onto a new line)
 %       \begin{subfigure}[c]{0.25\textwidth}
 %               \includegraphics[width=\textwidth]%{figs/ivec_cavg.png}
 %               \caption{Average cost}
  %              \label{fig:ivec_cavg}
  %      \end{subfigure}
  %      \caption{Results using utterance- and document-wise i-vectors calculated on PLP and MFCC features.}\label{fig:res_ivec}
%\end{figure}

\section{Conclusion}\label{sec:lid_conclusion}
In this section, two approaches to singing language identification were presented: One based on i-vector processing of audio features, and one based on the computation of phoneme statistics from posteriorgrams. In both cases, machine learning models were trained on the resulting data.\\

In the first approach, PLP, MFCC, and SDC features are extracted from audio data, and then run through an i-vector extractor. The generated i-vectors are then used as inputs for MLP and SVM training. The basic idea behind the i-vector approach is the removal of language-independent components of the signal. This effectively reduces irrelevance to the language identification tasks and also reduces the amount of training data massively.\\
The smallest data set is the \textit{NIST2003LRE} corpus. No feature configuration achieves good results when using the MLP backend. In this case, the small size of the corpus leads to overtraining. i-Vector processing only amplifies this problem by reducing the amount of data even further. The SVM backend, however, produces good results of up to 93\% for PLP features with i-vector extraction.\\
The \textit{OGIMultilang} corpus is a much bigger speech corpus. Training without i-vector extraction does not work well for any feature configuration. The best accuracy for this scenario was 68\%. Results of up to 83\% are achieved with i-vector processing. There is no large difference between SVM and MLP training, with SVMs having just a slight advantage.\\
Language identification for singing was expected to be a harder task than for speech due to the factors described in section \ref{sec:sota_speechtosinging}. The results on the \textit{YTAcap} corpus turn out to be somewhat worse than those for the \textit{OGIMultilang} corpus, which is of similar size. Once again, i-vector extraction improves the results from 63\% to 73\%.\\
The same experiment is repeated with no speaker overlap between training and test sets. The results fall significantly, indicating speaker influence on the model training. In a third experiment, the results are aggregated into documents by each speaker, which again leads to improved results. The best accuracy on \textit{OGIMultilang} is $92\%$, while on \textit{YTAcap}, it is $69\%$. Both experiments demonstrate that useful results can be obtained when limiting the task, e.g. by training on a set of known speakers or recording conditions, or by analyzing documents of longer durations. Alternatively, a wider range of speakers in the training data would lead to models that generalize better.\\
Overall, i-vector extraction reduces irrelevance in the training data and thereby leads to a more effective training. As additional benefits, the training process itself is much faster and less memory is used due to its data reduction properties. Most of the state-of-the-art approaches are based on PPRLM, which requires phoneme-wise annotations and a highly complex recognition system, using both acoustic and language models. In this respect, this system is easier to implement and merely requires language annotations.\\

The second presented method is a completely new language identification approach for singing. It is based on the output of various acoustic models, from which statistics are generated and SVM models are trained. In contrast to similar approaches for speech, no voice tokenization is performed. Since phoneme recognition on singing is not always reliable, the statistics are calculated directly on the phoneme posteriorgrams, although this does not take any temporal information into account. The acoustic models are trained only on English-language material (speech and singing); it would be interesting to test this with multi-language training data. Due to the statistics-based nature of the approach, it is not suited for language identification of very short audio recordings.\\
The accuracy of the result for singing is somewhat worse than the results obtained with the i-vector based approach. However, this new approach is much easier to implement and the feature vectors are shorter. For many applications, such posteriors need to be extracted anyway and can efficiently be used for language identification when long observations are available. The best accuracy of $63\%$ is obtained with acoustic models trained on the \textit{DampB} singing corpus.\\
Interestingly, the best result on the \textit{OGIMultilang} speech corpus is also obtained with these acoustic models (and is only 3 percent points below the one obtained with the i-vector approach). This possibly happens because the singing corpora provide a wider range of phoneme articulations. It would be interesting to try out these acoustic models for other phoneme recognition tasks on speech where robustness to varied pronunciations is a concern.




\chapter{Sung Keyword Spotting} \label{chap:kws}
%Keyword spotting is another task for which the new acoustic models described in chapter \ref{chap:phonerec} were employed. A keyword-filler HMM algorithm was selected due to its independence on phoneme durations, which are highly variable in singing as shown in section \ref{sec:sota_speechtosinging}. As described in section \ref{subsec:tech_kws}, keyword-filler HMMs consist of two sub-HMMs: One to model the keyword, and one to model everything else (= filler). The keyword HMM has a simple left-to-right topology with one state per keyword phoneme. The filler HMM is a fully connected loop of all phonemes. When the Viterbi path with the highest likelihood passes through the keyword HMM rather than the filler loop, the keyword is detected. Keyword detection was performed on whole songs, which is a realistic assumption for many practical applications. The \textit{ACAP} and \textit{DampTest} data sets were used for evaluation with the keyword set described in section \ref{sec:data_keywords}. Song-wise $F_1$ measures were calculated for evaluation.


In \cite{kws_overview}, three basic principles for keyword spotting in speech are mentioned:
\begin{description}
\item[LVCSR-based keyword spotting] {In this approach, a full transcription of the audio recording is performed using Large-Vocabulary Continuous Speech Recognition (LVCSR), which can then be searched for the keyword. 
%Both acoustic models and language models are employed for this step. The resulting text can then be searched for the requested keywords. 
This is expensive to implement and offers no tolerance for transcription errors - if the keyword is not transcribed correctly, it will never be found later.}
\item[Acoustic keyword spotting] {
%In contrast to the LVCSR approach, no full transcription is performed here. 
Acoustic KWS algorithms only search for the keyword on the basis of its acoustic properties. %This is commonly done with Hidden Markov Models (HMMs) and can either be performed directly on the audio or feature data, or on phoneme posteriorgrams obtained from an acoustic model.\\
This approach is easy to implement and provides some tolerance for pronunciation variations. However, it does not take any a-priori knowledge about the language into account (e.g. about plausible word or phoneme sequences).}
\item[Phonetic search keyword spotting]{
%Just like in LVCSR-based keyword spotting, 
Again, a full transcription of the audio recording is performed, but the full lattices are retained instead of just the final transcription. A phonetic search for the keyword can then be run on these lattices. This approach combines the a-priori knowledge of the LVCSR-based approach with the robustness of the acoustic approach.}
\end{description}

As described in section \ref{sec:sota_speechtosinging}, there are significant differences between speech and singing signals, which means that ASR approaches for keyword spotting cannot simply be transferred to singing. In particular, both LVCSR-based keyword spotting and Phonetic search keyword spotting depend heavily on predictable phoneme durations (within certain limits). When a certain word is pronounced, its phonemes will usually have approximately the same duration across speakers. The language model employed in both approaches will take this information into account. However, phoneme durations in singing are not as predictable in speech, as figure \ref{fig:phoneme_stats} demonstrates. For this reason, a simpler acoustic approach using keyword-filler HMMs is employed in this work.\\
Keyword-filler HMMs have been described in \cite{szoeke} and \cite{jansen}. In general, two separate HMMs are created: One for the requested keyword, and one for all non-keyword regions (=filler). The keyword HMM has a simple left-to-right topology with one state per keyword phoneme, while the filler HMM is a fully connected loop of states for all phonemes. These two HMMs are then joined. Using this composite HMM, Viterbi decoding is performed on the phoneme posteriorgrams. Whenever the Viterbi path passes through the keyword HMM, the keyword is detected. The likelihood of this path can then be compared to an alternative path through the filler HMM, resulting in a detection score. A threshold can be employed to only return highly scored occurrences. Additionally, the parameter $\beta$ can be tuned to adjust the model. It determines the likelihood of transitioning from the filler HMM to the keyword HMM. The whole process is illustrated in figure \ref{fig:kf_hmm}.

\begin{figure}
 \centerline{\framebox{
 \includegraphics[width=.8\textwidth]{images/kw_filler_hmm.png}}}
 \caption{Keyword-filler HMM for the keyword ``greasy" with filler path on the left hand side and two possible keyword pronunciation paths on the right hand side. The parameter $\beta$ determines the transition probability between the filler HMM and the keyword HMM. \cite{jansen}}
 \label{fig:kf_hmm}
\end{figure}
Integration with the phoneme recognition system is shown in figure \ref{fig:process_training_kws}. Effectively, the keyword-filler HMM is added as a post-processing step after classification, and thus performed on the posteriorgrams.
\begin{figure}
	\begin{center}
		\includegraphics[width=0.8\textwidth]{images/process_classification_kws.png}
		\caption{Schematic of the procedure for keyword spotting.}
		\label{fig:process_training_kws}
	\end{center}
\end{figure}

Keyword spotting results are considered correct when they are detected within the expected songs, and are evaluated according to the $F_1$ measure. This measure is the harmonic mean of precision $P$ and recall $R$:
 \begin{equation}
    F_1 = 2 \frac{P \cdot R}{P + R}
 \end{equation}  
This measure is especially suited for cases where the classes are not balanced; in keyword spotting, occurrence of a keyword is much rarer than non-occurrence. In continuous speech recognition, the Figure Of Merit measure is often used \cite{fom}; however, since timing is not an issue in many applications for sung keyword spotting, this measure is not employed here.\\

Keyword detection was performed on whole songs, which is a realistic assumption for many practical applications. The \textit{ACAP} and \textit{DampTest} data sets were used for evaluation with the keyword set described in section \ref{sec:data_keywords}. Song-wise $F_1$ measures were calculated for evaluation. As in the phoneme recognition experiments, no cross validation is employed because the training and test data sets serve different purposes.


\section{Keyword spotting using keyword-filler HMMs} \label{sec:kws_kwfhmm}
\subsection{Comparison of acoustic models}
Phoneme posteriorgrams were generated with the various acoustic models described in section \ref{sec:phonerec_acap}. The results in terms of $F_1$ measure across the whole \textit{DampTest} sets are shown in figure \ref{fig:kws_exp1}. Figure \ref{fig:kws_exp1_acap} shows the results of the same experiment on the small \textit{ACAP} data set.

Across all keywords, a document-wise $F_1$ measure of $0.44$ is obtained using the posteriorgrams generated with the \textit{TIMIT} model on the \textit{DampTest} data sets. This result remains the same for the \textit{TimitM} models trained on ``songified'' speech. In this experiment, using models trained on the \textit{DAMP}-based singing data sets only improves the results marginally, with $F_1$ measures of $0.47$ for the \textit{DampB} model, and $0.46$ with the much smaller \textit{DampBB\_small} model. Surprisingly, in this case, the model trained on the medium-size balanced data set \textit{DampBB} performs a little worse than the smallest one; however, this might just be due to some statistical fluctuation. In general, results on these test data sets are inconclusive. There are several reasons for this: First, the annotations were generated automatically and the keywords were picked from the aligned lyrics. The singers do not always perform or pronounce them correctly. Additionally, the keyword approach can be tuned easily for high recall; then, the precision becomes the deciding factor for $F_1$ calculation. Considering the size of the data set, keyword occurrences are relatively rare, which makes obtaining a high precision more difficult and blurs the $F_1$ measures between approaches.\\

On the hand-annotated \textit{ACAP} test set, the differences are somewhat more pronounced (but still not statistically significant). The $F_1$ measure is $0.48$ for the \textit{TIMIT} model, and rises to $0.52$ with the \textit{DampB} model. The \textit{TimitM} and \textit{DampBB}�models both produce $F_1$ measures of $0.49$. The higher over-all values are caused by the more accurate annotations and by the higher-quality singing. Additionally, the data set is much smaller with fewer occurrences of each keyword, which emphasizes both positive and negative tendencies in the detection.\\

In general, recalls are usually close to $1$, and precisions often in the range of $0.2$ to $0.5$ (with much lower and higher outliers). For this reason, an approach that could exploit a configuration with high recalls and then discard unlikely occurrences may offer an improvement. This idea is explored further in section \ref{sec:kws_duration}.

\begin{figure}
        \centering
        \begin{subfigure}[t]{0.4\textwidth}
		 \includegraphics[width=\textwidth]{images/kws_exp1.png}
                \caption{\textit{DampTest}}
                \label{fig:kws_exp1}

        \end{subfigure}%
         %add desired spacing between images, e. g. ~, \quad, \qquad, \hfill etc.
          %(or a blank line to force the subfigure onto a new line)
        \begin{subfigure}[t]{0.4\textwidth}
                \includegraphics[width=\textwidth]{images/kws_exp1_acap.png}
                \caption{\textit{ACAP}}
                \label{fig:kws_exp1_acap}
        \end{subfigure}
        \caption{$F_1$ measures for keyword spotting results using posteriorgrams generated with various acoustic models (error bars represent standard error over keywords).}
          \end{figure}
%\vspace{-5px}

\subsection{Gender-specific acoustic models}
%TODO: ACAP??
Keyword spotting was also performed on the posteriorgrams generated with the gender-dependent models trained on \textit{DampF} and \textit{DampM} (also described in section \ref{sec:phonerec_acap}). The results are shown in figure \ref{fig:kws_exp2}.

Similar to the phoneme recognition results from Experiment C, the gender-dependent models offer no improvements over the mixed-gender ones of the same size, and are in the same range as the one trained on much more data (\textit{DampB}). The $F_1$ measures for the female test set are $0.48$ for the \textit{DampB} model, and $0.47$ for both the \textit{DampBB} and the \textit{DampFB} model. For the male test set, they are $0.47$ for the \textit{DampB}�model, and $0.45$ for the other two.



\subsection{Individual analysis of keyword results}
%TODO: ACAP??
Figure \ref{fig:kws_exp3} shows the individual $F_1$ measures for each keyword using the best model (\textit{DampB}), ordered by their occurrence in the \textit{DampTest} sets from high to low (i.e. number of songs which include the keyword). There is a tendency for more frequent keywords to be detected more accurately. This happens because a high recall is often achievable, while the precision depends very much on the accuracy of the input posteriorgrams. The more frequent a keyword, the easier it also becomes to achieve a higher precision for it.

As shown in literature \cite{phdthesis:thambiratnam}, the detection accuracy also depends on the length of the keyword: Keywords with more phonemes are usually easier to detect. This explains the relative peak for ``every" in contrast to ``think" or ``night". Since keyword detection systems tend to perform better for longer words and most of the keywords only have 3 or 4 phonemes, the results achieved so far are especially interesting.

One potential source of confusion are sequences of phonemes that overlap with keywords, but are not included in the calculation of the precision. Identically spelled parts of words were included, but split phrases and different spellings were not (e.g. ``away" as part of ``castaway" would be counted, but ``a way" would not be counted as ``away"). This lowers the results artificially and could be an avenue for future improvement. Additionally, only one pronunciation for each keyword was provided, but there may be several possible.

\begin{figure}
 \begin{center}
                \includegraphics[width=0.5\textwidth]{images/kws_exp2.png}
                \caption{$F_1$ measures for keyword spotting results on the \textit{DampTestM} and \textit{DampTestF} data sets using mixed-gender and gender-dependent models (error bars represent standard error over keywords).}
                \label{fig:kws_exp2}
                 \end{center}
 \end{figure}

\begin{figure}
 \begin{center}
% 	\begin{subfigure}[t]{0.4\textwidth}
                \includegraphics[width=.6\textwidth]{images/kws_exp3.png}
%               \caption{\textit{DampTest}}
%                \label{fig:kws_exp3_damp}
%       	\end{subfigure}
% 	\begin{subfigure}[t]{0.4\textwidth}
% 		\includegraphics[width=\textwidth]{images/kws_exp3_acap.png}
% 		\caption{\textit{ACAP}}
% 		\label{fig:kws_exp3_acap}
% 	\end{subfigure}
       	
     	 \caption{Individual $F_1$ measures for the results for each keyword, using the acoustic model trained on \textit{DampB}.} 		\label{fig:kws_exp3}
                 \end{center}
 \end{figure}
%\vspace{-5px}





\section{Keyword spotting using duration-informed keyword-filler HMMs}\label{sec:kws_duration}
\subsection{Approach}
As mentioned above, a high recall is easily achievable with the described approach, but the comparatively low precision decreases the over-all result. Therefore, using side information to reject false positives would be a helpful next step.\\
One such source of information are the durations of the detected phonemes. As shown in figure \ref{fig:phoneme_stats}, each phoneme in the \textit{TIMIT} speech database has a fairly fixed duration. In singing, the vowels' durations vary a lot, but the consonants' are still quite predictable. Standard HMMs do not impose any restrictions on the state durations, resulting in a geometric distribution which does not correspond to naturally observed distributions of phoneme durations.\\
As first described in \cite{ferguson}, introducing restrictions on state durations can improve the recognition results. In \cite{juang}, Juang et al. present two basic approaches for duration modeling in HMMs: Internal duration modeling and Post-processor duration modeling.\\

In both approaches, parametric state duration models for each phoneme need to be calculated first \cite{levinson}. Several distributions have been tested for this task (e.g. Gaussian ones), but Burshtein showed that gamma distributions are best at modeling naturally occurring phoneme duration distributions \cite{burshtein}:
\begin{equation}\label{eq:d}
d(\tau) = K \exp \{ - \alpha \tau \} \tau^{p-1}
\end{equation}
where $\tau = 0,1,2,...$ are the possible state durations in frames and $K$ is a normalizing factor. The parameters $\alpha$ and $p$ are estimated according to
\begin{equation}
\hat{\alpha} = \frac{E \{\tau\}}{VAR\{\tau\}} , \hat{p} = \frac{E^{2} \{\tau\}}{VAR\{\tau\}}
\end{equation}
where $E$ is the distribution mean and $VAR$ is the distribution variance. An example is shown in figure \ref{fig:distributions}. In this work, $E$ and $VAR$ are estimated from the phonetically annotated data sets.

\begin{figure}
 \begin{center}
                \includegraphics[width=0.7\textwidth]{images/distributions.png}
                \caption{An example of the empiric duration distribution of one phoneme state (solid line) and three approximations: Gaussian (dashed), geometric(dash-dot), and gamma (dotted). \cite{burshtein}}
                \label{fig:distributions}
                 \end{center}
 \end{figure}

In internal duration modeling, the durations are incorporated directly into the Viterbi alignment. This means that the Viterbi output will already be a state sequence that is optimal with regards to the a-priori phoneme duration knowledge. It is, however, computationally expensive. In previous experiments \cite{kruspe_kws2}, this approach did not produce better results than the much easier to implement post-processor duration modeling. Therefore, this section will focus on that approach.\\

When using post-processor duration modeling, knowledge about plausible phoneme durations is imposed on the result of the Viterbi alignment, the obtained state sequence. This is computationally cheap, but only results in a new likelihood score for the calculated sequence and does not provide better possible state sequences. As described in \cite{juang}, the state sequence obtained from the Viterbi alignment can be re-scored according to:
\begin{equation} \label{eq:pp_ll}
 \log \hat{f} = \log f + \gamma \sum_{k=1}^{N} d_{k}(\tau_k) 
\end{equation}
where $f$ is the original likelihood of the sequence, $\gamma$ is a weighting factor, $k=1...N$ are the discrete states in the state sequence, $\tau_k$ are their durations, and $d_{k}(\tau_k)$ is, again, the probability of state $k$ being active for the duration $\tau_k$.\\
Using keyword-filler HMMs, only one state sequence per utterance is obtained, which either contains the keyword or not. It is therefore not possible to compare these likelihood scores and equation \ref{eq:pp_ll} cannot be applied directly. To still be able to integrate post-processor duration modeling, the HMM parameters are tuned to obtain a high recall value. Then, the duration likelihood (second half of equation \ref{eq:pp_ll}) is calculated for all found occurrences of the keyword and normalized by the number of states taken into account:
\begin{equation}
 dl =  \frac{1}{N} \sum_{k=1}^{N} d_{k}(\tau_)
\end{equation}
Then all occurrences where $dl$ is below a certain threshold are discarded.\\

For the presented results, duration statistics from the \textit{ACAP} data set were used, and only the consonants' durations were taken into account (since vowel durations vary much more as shown in figure \ref{fig:phoneme_stats}). However, additional experiments showed that the result only varies slightly when using speech statistics instead, and when also discarding unlikely vowel durations. This probably happens because the keywords do not contain many states anyway, and because the duration distribution for vowels has a large variance, allowing for a wide range of durations.

\subsection{Results}

\begin{figure}
        \centering
        \begin{subfigure}[c]{0.4\textwidth}
		\includegraphics[width=\textwidth]{images/kws_exp4_acap.png}
                \caption{$F_1$ measures for keyword spotting results on the \textit{ACAP} data set with post-processor duration modeling.}
                \label{fig:kws_exp4_acap}
                
        \end{subfigure}%
         %add desired spacing between images, e. g. ~, \quad, \qquad, \hfill etc.
          %(or a blank line to force the subfigure onto a new line)
        \begin{subfigure}[c]{0.3\textwidth}
                \includegraphics[width=\textwidth]{images/kws_exp4.png}
                \caption{$F_1$ measures for keyword spotting results on the \textit{DampTest} data sets with post-processor duration modeling.}
                \label{fig:kws_exp4}
                
        \end{subfigure}
        \caption{$F_1$ measures for keyword spotting results using posteriorgrams generated with various acoustic models with post-processor duration modeling (error bars represent standard error over keywords).}
          \end{figure}


Results with and without post-processor duration modeling for the \textit{ACAP} data set are shown in figure \ref{fig:kws_exp4_acap}. The same acoustic models as in the previous experiment were tested. As these results show, $F_1$ measures improve for all configurations when post-processor duration modeling is employed (significant with $p<0.1$). The effect is somewhat stronger for the \textit{DAMP} models than for the \textit{TIMIT} model (significant with $p<0.01$). The best result rises from $0.52$ to $0.61$ with the \textit{DampB} model. Analysis of the detailed results shows that the precision can be improved considerably when detected occurrences with implausible phoneme durations are discarded. However, this often also decreases the recall, resulting in the shown $F_1$ results.\\

The approach was also tested on the \textit{DampTest} data sets for two acoustic models. $F_1$ measures on these data sets are generally blurry for the reasons described in section \ref{sec:kws_kwfhmm}. In this case, the results are just a little bit higher with post-processor duration modeling.


%TODO: clean up
\section{Conclusion}\label{sec:kws_conclusion}
%insgesamt besser
%huge DampB set best, but 3% DampBB almost as good
% even v small DampBB_small better than Timit
% context does not help
% gender models slightly better for kws, not better for phone rec --> variability?
% results phonerec
% results kws - esp. interesting because short
%---auto alignment error source


In this chapter, an approach for keyword spotting using the new acoustic models trained on singing was described. Keyword spotting is performed by extracting phoneme posteriorgrams generated with these new models from the audio, and then running them through a keyword-filler HMM to detect 15 keywords. On the \textit{DampTest} data sets, the resulting $F_1$ measure rises from $0.44$ for the models trained on speech (\textit{TIMIT}) to $0.47$ for the new models. In general, results on the \textit{DampTest} data sets are inconclusive because the effect of the different models is shadowed by issues with the test data itself - i.e. automatic and thus possibly inaccurate annotations, amateur singing, and a relatively low frequency of keywords because of the large size of the data sets. On the smaller, hand-annotated \textit{ACAP} test set, the results become clearer: The best $F_1$ measure for the models trained on \textit{TIMIT} is $0.48$, and $0.52$ with those trained on \textit{DampB}.\\
This result is especially interesting because most of the keywords have few phonemes. Gender-dependent models perform similar to mixed-gender models of the same size. Individual analysis of the keyword results shows that keywords that occur more frequently are detected more accurately. This probably happens because the approach is able to obtain high recall easily, but precision is an issue. The more frequent a keyword, the easier obtaining higher precisions becomes. Additionally, keywords with more phonemes are detected more accurately than short ones because there is more information to base detection on.\\

One idea to improve precision was using additional information to discard implausible detections. This idea was tested in a second approach by integrating knowledge about phoneme durations. Means and variances of phoneme durations are calculated from annotated data (in particular, the \textit{ACAP} singing data set), and then occurrences with phoneme durations outside of their gamma distributions were ignored. This approach improved $F_1$ measures by up to $9$ percent points on the \textit{ACAP} test data, with the best result being $0.61$. On the \textit{DampTest} data sets, the effect is still existent, but not very pronounced. This is probably because of the blurriness of the result described above.\\

This approach was only tested with MFCC features. As preliminary experiments suggest \cite{kruspe_kws1}, other features like TRAP or PLP may work better on singing. So-called log-mel filterbank features have also been used successfully with DNNs \cite{hinton}. Another interesting factor is the size and configuration of the classifiers, of which only one was tested (after a small grid search to validate this choice).\\
As in the phoneme recognition experiments, there is not much of a difference between the acoustic model trained on the more than 6,000 songs of the \textit{DAMP} data set and the one trained on only $4\%$ of this data. It would be interesting to find the exact point at which additional training data does not further improve the models. On the evaluation side, a keyword spotting approach that allows for pronunciation variants or sub-words may produce better results. Language modeling might also help to alleviate some of the errors made during phoneme recognition.\\
These models have not yet been applied to singing with background music, which would be interesting for practical applications. Since this would probably decrease the result when used on big, unlimited data sets, specialized systems would be more manageable, e.g. for specific music styles, sets of songs, keywords, or applications. Searching for whole phrases instead of short keywords could also make the results better usable in practice.\\
As shown in \cite{mesaros_alignment} and \cite{goto_alignment} and in the next chapter, alignment of textual lyrics and singing already works well. A combined approach that also employs textual information could be very practical.


\chapter{Applications} \label{chap:applications}
\section{Experiments on the QMUL Expletive data set}
\subsection{Implementation}
\subsection{Results}
\section{Experiments on the the "69 Love Songs" data set}
\subsection{Implementation}
\subsection{Results}
\section{Automatic lyrics retrieval}
\subsection{Implementation}
\subsection{Results}
\chapter{Conclusion} \label{chap:conclusion}
\section{Summary}
In this work, ASR algorithms for various tasks were applied and adapted to singing. Five such tasks were considered: Phoneme recognition, language identification, keyword spotting, lyrics-to-audio alignment, and lyrics retrieval.\\

Two strategies for improving speech recognition technologies for singing in general were identified: Training better-matching acoustic models for phoneme recognition, and making specific algorithms more robust to singing - either by balancing possible deficits that occur because of the singing-specific characteristics, or by exploiting knowledge about sung vocal production.\\

The main bottleneck in almost all tasks is the \textbf{recognition of phonemes} in singing. Three approaches for this were tested: Recognition with models trained on speech, recognition with models trained on speech that was made more ``song-like'' (``songified''), and recognition with models trained on singing. The main issue in this area of research is the lack of large training data sets of singing with phoneme annotations. For this reason, the ``songified'' approach was developed, which slightly improved results over the models trained on pure speech when applied to sung recordings. For training on actual singing data, the large \textit{DAMP} data set of unaccompanied amateur singing recordings was chosen. Phonemes obtained from the matching textual lyrics were automatically aligned to these recordings, and new acoustic models were trained on the resulting annotated data set. These models showed large improvements for phoneme recognition in singing.\\\\
This training strategy is the main contribution of this work, and forms the basis for all the subsequent tasks. As proposed in the introduction, various algorithms for the individual tasks were then tested and adapted by improving robustness when applied to singing instead of speech, or by taking useful knowledge about singing into account. Robustness was, for example, increased by using i-vector extraction for language identification, by employing keyword-filler HMMs to keyword spotting, and by adapting alignment approaches to enable varying phoneme durations. Singing characteristics are exploited, for example, by basing language identification methods on long recordings, by integrating phoneme duration knowledge into keyword spotting, and by including known phoneme confusions in singing into the alignment and retrieval algorithms.\\
The following paragraphs will sum up the individual adaptations to the tasks in detail:\\
%This training strategy is the main contribution of this work, and forms the basis for all the subsequent tasks. In addition, various singing-specific adaptations were performed for individual ASR problems in singing. These will be summed up in the following paragraphs:\\

For \textbf{language identification} in singing, the state-of-the-art i-vector approach from ASR was tested and produced good results. It is especially suited to singing because it implements a strategy for removing irrelevant and repetitive information, such as signal components caused by the channel. These results improve to a level that is usable in practical applications when long recordings are available, which is usually the case for singing. In addition to this method, a new approach based on phoneme statistics was developed. In this approach, phoneme posteriorgrams are first generated with the models trained in the phoneme recognition experiments, then statistics are calculated, and different models are trained on those. This approach does not perform quite as well as the i-vector method, but is easier to implement under certain circumstances (when phoneme recognition needs to be performed anyway). Long recordings are necessary to calculate meaningful statistics, but, once again, this is not a problem on singing data.\\

\textbf{Keyword spotting} for singing is a task that has not frequently been a subject of research, and there is still a lot of room for improvement. In this work, keyword-filler HMMs were tested. In contrast to other state-of-the-art methods, this approach does not rely on highly stable phoneme durations, which are very common in speech, but not in singing. The method detects keywords with high recall, but often poor precision. Knowledge about probable phoneme durations was obtained from analyzing sung recordings and added in a post-processing step. This improved results for frequently-occurring keywords. The performance may not be good enough for practical applications yet, but these approaches show promise for future research. The length of the keyword (number of phonemes) is a major deciding factor. Searching for long keywords or phrases is already possible in practice, and this is a probable usage scenario for music collections.\\

\textbf{Lyrics-to-audio alignment} is a relatively well-researched topic. In this work, the new acoustic models were also applied to this task, alongside a traditional HMM-based approach. Two methods for utilizing phoneme posteriorgrams for alignment were developed. The first one calculates a DTW between the posteriorgram and a binary representation of the expected lyrics. The resulting path is the alignment. Since no knowledge about the duration of the constituting phonemes is available in the text, and these durations vary widely in singing, the algorithm is modified to not punish such variations. The second approach first extracts a plausible phoneme sequence from the posteriorgram, taking known phoneme confusions in singing into account and compressing extended productions of similar phonemes. Then, the Levenshtein distance between this symbolic (phonetic) representation and the phonemes obtained from the textual lyrics is calculated, and the alignment path is the result.\\
The HMM and DTW approaches perform acceptably, while the Levenshtein approach produces much better results. These algorithms are suited for practical applications. One such example was described: The automatic detection and removal of expletives in songs.\\

\textbf{Lyrics retrieval} from a textual database based on sung queries is performed with the same algorithms as the alignment. Alignment is performed between the posteriorgram generated from the audio and all possible lyrics. Then, the ones with the best scores/lowest distances are selected as the result.\\
Once again, the DTW approach performs feasibly, while the Levenshtein approach produces very good results that are practically usable, even on short queries. Additionally, it allows for fast search and optimizations on large lyrics databases, which is a so-far underresearched use case.

\section{Contributions}


Major research contributions of this work include:

\paragraph{A new large data set of phonetic annotations for unaccompanied singing}
As described in sections \ref{sec:data_damp} and \ref{sec:phonerec_acap}, the pre-existing \textit{DAMP} data set of unaccompanied singing was used as a basis for this. The matching textual lyrics were scraped from the internet, and automatically aligned to the audio with an HMM model trained on speech. Various strategies were tested for this. The resulting phoneme annotations are not 100\% accurate, but the data set is very large and these inaccuracies balance out, as the experiments building on the data demonstrate. Various configurations for this data set were composed for different purposes, including female and male singing test sets.

\paragraph{Acoustic models trained on this data set}
New acoustic models (DNNs) were then trained on the generated training data sets (section \ref{sec:phonerec_acap}). The performance of these models was thoroughly compared to others, and they generally performed better when applied to singing phoneme recognition. They were also integrated into the other tasks (language identification, keyword spotting, lyrics-to-audio alignment and retrieval) and demonstrated improvements in almost all cases.
%+evaluation

\paragraph{A novel approach for language identification based on phoneme statistics}
In addition to an i-vector-based approach from ASR, a completely new method for language identification was developed on the basis of phoneme posteriorgrams. This approach is particularly suited to singing because long recordings are often available in music applications. As described in section \ref{sec:lid_stats}, phoneme statistics are calculated on these posteriorgrams over long time frames (e.g. whole songs), and then new models are trained on these statistics. Unseen recordings can then be subjected to the same process to determine their languages. The results of this approach were not quite as good as those of the i-vector algorithm, but they show promise. This method is easier to implement than the one using i-vectors when long audio sequences are available, and when phoneme posteriorgrams need to be extracted from them for other purposes.

\paragraph{A new method for flexibly integrating knowledge about phoneme durations into keyword spotting}
The presented approach for keyword spotting is based on keyword-filler HMMs. In order to improve its results, the algorithm was expanded to take a-priori knowledge about the plausible durations of individual phonemes into account (see section \ref{sec:kws_duration}). This was implemented via post-processor duration modeling: The model is tuned to return many results (resulting in a high recall), and then those with improbable phoneme durations are discarded to improve the precision. Influence of individual phoneme durations is easily turned on or off.  This form of duration modeling has not been applied to keyword-filler HMMs before. 

\paragraph{A novel method for extracting plausible phoneme sequences from posteriorgrams}
The described phoneme recognition approaches result in phoneme posteriorgrams, and many algorithms operate directly on them. However, there are also applications where a fixed phoneme sequence is required. In section \ref{sec:retrieval_phone}, a new method for extracting such sequences from posteriorgrams is presented. Traditionally, HMMs would be used for this task, but they have a number of disadvantages in the use case at hand. The results of this phoneme extraction method are further used in this work for alignment and retrieval.

\paragraph{Two new approaches for lyrics-to-audio alignment}
In addition to classic HMM-based lyrics-to-audio alignment, this thesis presents two novel approaches that operate on posteriorgrams: One based on DTW, and one based on Levenshtein distance calculation. This enables them to make use of the new acoustic models. The developed algorithms were submitted to the \textit{MIREX} 2017 challenge for lyrics-to-audio alignment, where the Levenshtein-based approach outperformed the other submissions.

\paragraph{Two first approaches to lyrics retrieval based purely on sung queries}
The same two approaches for lyrics-to-audio alignment can be applied to the task of lyrics retrieval. This is an application that has only rarely been the subject of research so far. This thesis describes the implementation of the presented methods for retrieval of textual song lyrics from a database with sung queries.


\section{Limitations and future work}
%applications - genre, mood, similarity
% robustness for asr tasks
%other languages
%polyphonic

Suggestions for the individual tasks and approaches are described in the corresponding chapters. This section focuses on the over-all field of ASR for singing.\\
In general, phoneme recognition is still the major bottleneck for many of the described tasks. This thesis shows possible ways to improve this recognition. Future research approaches may employ even larger data sets of singing for training, preferably with more reliable phonetic annotations (e.g. manual annotations). Alternatively, models could be trained in an unsupervised or semi-supervised way with large amounts of unannotated data and small sets of reliably annotated data. New machine learning techniques could be tested as well, such as CNNs, RNNs, or end-to-end models, or methods for unsupervised pre-training like autoencoders.\\
Unusual or unclear pronunciations, accents, and unusual voices (e.g. children's voices) also lead to unsatisfactory results in many tasks. New models trained on a larger variety of data would improve robustness. The same applies to problems with the audio quality or channels, and recording conditions not seen during training.\\

Some of the algorithms require exact annotations (e.g. alignment and retrieval). This is not always a given in a real-world scenario, where singers will perform different words, additional vocalizations, unexpected repetitions etc. In a future step, it is possible to make the developed algorithms more robust to such changes by loosening constraints on the expected words or allowing loops in the sequence of words.\\

Most of the presented approaches have so far only been applied to unaccompanied singing. A major step forward would be the adaptation to polyphonic music. This was already done for alignment and retrieval, but could be improved significantly. There are several possible routes that future research could take. First, acoustic models could be trained on accompanied singing instead. This would require large amounts of realistic training data, and probably more sophisticated models in order to represent the more complex structures. Second, source separation could be integrated to extract the singing track from the audio, and only perform the analysis on this part. As a third alternative, Vocal Activity Detection could be applied beforehand to only analyze the segments where actual singing occurs. In the presented experiments, instrumental solos in particular frequently led to misalignments or recognition errors.\\

Finally, the developed algorithms could be employed for the applications described in section \ref{sec:intro_motivation}. It would be highly interesting to see them integrated into other MIR systems, e.g. for genre or regional classification or for mood detection. The other practical scenarios could also make good use of them, for example in karaoke systems, in audio identification, or in cover song detection. Search algorithms or similarity calculation based on the analyzed characteristics could be used in practical MIR systems.

\chapter{Future work} \label{chap:future}



%TODO: contributions...?

% Literaturverzeichnis
\addcontentsline{toc}{chapter}{Bibliography}
\cleardoublepage
\footnotesize
\ihead[]{Bibliography}
\bibliographystyle{IEEEbib}
\bibliography{literatur} 
\normalsize

% Abbildungsverzeichnis
\addcontentsline{toc}{chapter}{List of Figures}
\cleardoublepage
\ihead[]{List of Figures}
\listoffigures

% Tabellenverzeichnis
\addcontentsline{toc}{chapter}{List of Tables}
\cleardoublepage
\ihead[]{List of Tables}
\listoftables

% Abkürzungsverzeichnis
%\addcontentsline{toc}{chapter}{List of Abbreviations and Symbols}
%\chapter*{Abbreviations}

\cleardoublepage
\ihead[]{List of Abbreviations and Symbols}
\addcontentsline{toc}{chapter}{List of Abbreviations and Symbols}
\renewcommand{\nomname}{List of Abbreviations and Symbols}
\setlength{\nomlabelwidth}{.20\hsize}
\renewcommand{\nomlabel}[1]{#1 \dotfill}
% Zeilenabstnde verkleinern
\setlength{\nomitemsep}{-\parsep}
\markboth{\nomname}{\nomname}

\nomenclature{$A_C$}{Class-Wise Accuracy}
\nomenclature{$A_M$}{Multi-Class Accuracy}
\nomenclature{ACF}{Auto-Correlation Function}
\nomenclature{ASE}{Audio Spectral Envelope}
\nomenclature{BPM}{Beats per minute}
\nomenclature{CFS}{Correlation-Based Feature Selection}
\nomenclature{CUIDADO}{Content-based  Unified  Interfaces and Descriptors for  Audio/Music  Databases  available  Online}
\nomenclature{DCT}{Discrete Cosine Transform}
\nomenclature{EPCP}{Enhanced Pitch Class Profile}
\nomenclature{ETD}{Equal-Tempered Deviation}
\nomenclature{$F$}{$F$ measure}
\nomenclature{FFT}{Fast Fourier Transform}
\nomenclature{FIR}{Finite Impulse Response}
\nomenclature{Fraunhofer IDMT}{Fraunhofer Institute for Digital Media Technology}
\nomenclature{GMM}{Gaussian Mixture Model}
\nomenclature{H2A}{Harmonic-to-Attack Ratio}
\nomenclature{HMM}{Hidden Markov Model}
\nomenclature{Hz}{Hertz}
\nomenclature{IRMFSP}{Inertia Ratio Maximization using Feature Space Projection}
\nomenclature{ITU-T}{International Telecommunication Union, Telecommunication Standardization Sector} 
\nomenclature{kbps}{Kilobit per second}
\nomenclature{KEFIR}{Knowledge Expert Framework for Information Retrieval}
\nomenclature{KNN}{$k$-Nearest Neighbour (Classifier)}
\nomenclature{LDA}{Linear Discriminant Analysis}
\nomenclature{MARSYAS}{Music Analysis, Retrieval and Synthesis for Audio Signals}
\nomenclature{MFCC}{Mel-Frequency Cepstral Coefficient(s)}
\nomenclature{MIDI}{Musical Instrument Digital Interface}
\nomenclature{MIR}{Music Information Retrieval}
\nomenclature{MIREX}{Music Information Retrieval Evaluation eXchange}
\nomenclature{MPEG}{Moving Picture Experts Group}
\nomenclature{NN}{Neural Networks}
\nomenclature{$P$}{Precision}
\nomenclature{PCA}{Principal Component Analysis}
\nomenclature{$R$}{Recall}
\nomenclature{RAS}{Rhythmic Auditory Stimulation}
\nomenclature{RBF}{Radial Basis Function}
\nomenclature{SCM}{Spectral Crest Measure}
\nomenclature{SFM}{Spectral Flatness Measure}
\nomenclature{SVM}{Support Vector Machine}
\nomenclature{UNESCO}{United Nations Educational, Scientific and Cultural Organization}
\nomenclature{ZCR}{Zero-Crossing Rate}


% \printglossary
\printnomenclature

% Anhang
%%
%% Anhang
%%
\cleardoublepage
\ihead[]{Appendix}
\ohead[]{\pagemark}
\begin{appendix}
\chapter{Appendix}

A lot of stuff that didn't fit into the main part ...


% Selbstständigkeitserklärung
\chapter{Eigenst�ndigkeitserkl�rung}

Die vorliegende Arbeit habe ich selbstst�ndig ohne Benutzung anderer als der angegebenen Quellen angefertigt. 

Alle Stellen, die w�rtlich oder sinngem�� aus ver�ffentlichten Quellen entnommen wurden, sind als solche deutlich kenntlich gemacht. Die Arbeit ist in gleicher oder �hnlicher Form oder auszugsweise im Rahmen einer oder anderer Pr�fungen noch nicht vorgelegt worden.
\\[2cm]
Ilmenau, 17.12.2013
\\[1cm]


Sheldon Cooper



 





% Thesen
\chapter*{Thesis Summary}
\thispagestyle{empty}

\begin{enumerate}

\item Scissors cuts paper, paper covers rock, rock crushes lizard, lizard poisons Spock, Spock smashes scissors, scissors decapitates lizard, lizard eats paper, paper disproves Spock, Spock vaporizes rock, and as it always has, rock crushes scissors.

\item I'm not insane, my mother had me tested!

\item All I need is a healthy ovum and I can grow my own Leonard Nimoy!

\end{enumerate}
\newpage
\section*{Thesen}
\thispagestyle{empty}
\begin{enumerate}
\item These 1
\item These 2
\item These 3

\end{enumerate}

\end{appendix}

\end{document}

