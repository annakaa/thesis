%\begin{otherlanguage}{english}
\begin{abstract}
The research field of Music Information Retrieval is concerned with the automatic analysis of musical characteristics. One of its goals in particular is the extraction of knowledge about musical pieces from the audio signal. One aspect that has not received much attention so far is the automatic analysis of sung lyrics.\\
On the other hand, the field of Automatic Speech Recognition has produced many methods for the automatic analysis of speech. These approaches, however, have rarely been applied to singing so far.\\
This thesis analyzes the feasibility of applying various speech recognition methods to singing, and suggests adaptations for this different type of signal. In addition, the routes to practical applications for these systems are described. Five tasks are considered: Phoneme recognition, language identification, keyword spotting, lyrics-to-audio alignment, and retrieval of lyrics from sung queries.\\
For the task of phoneme recognition, two new large data sets are created. The first one is generated by making speech recordings more ``song-like''. The second one is based on an existing singing data set with matching textual lyrics. An approach is presented to automatically align these lyrics to the audio to be able to use the resulting data set for other tasks. Both new data sets are thoroughly tested for training phoneme models. The second data set offers large improvements compared to previous models trained on speech.\\
For the keyword spotting tasks, 
\end{abstract}
%\end{otherlanguage}
%\begin{otherlanguage}{ngerman}
\renewcommand{\abstractname}{Kurzfassung}
\newcommand{\dtabstract}{\hyphenpenalty=10000}
{\dtabstract
\begin{abstract}
Das Higgs-Teilchen geh�rt zum Higgs-Mechanismus, einer schon in den 1960er Jahren vorgeschlagenen Theorie, nach der alle fundamentalen Elementarteilchen (beispielsweise das Elektron) ihre Masse erst durch die Wechselwirkung mit dem allgegenw�rtigen Higgs-Feld erhalten. Als einziges Teilchen des Standardmodells ist das Higgs-Boson experimentell noch nicht vollst�ndig gesichert.
\end{abstract}
}
%\end{otherlanguage}
