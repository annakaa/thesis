%\begin{otherlanguage}{english}
\begin{abstract}
The Higgs boson or Higgs particle is an elementary particle initially theorised in 1964,[6][7] and tentatively confirmed to exist on 14 March 2013.[8] The discovery has been called "monumental"[9][10] because it appears to confirm the existence of the Higgs field,[11][12] which is pivotal to the Standard Model and other theories within particle physics. In this discipline, it explains why some fundamental particles have mass when the symmetries controlling their interactions should require them to be massless, and?linked to this?why the weak force has a much shorter range than the electromagnetic force.
\end{abstract}
%\end{otherlanguage}
%\begin{otherlanguage}{ngerman}
\renewcommand{\abstractname}{Kurzfassung}
\newcommand{\dtabstract}{\hyphenpenalty=10000}
{\dtabstract
\begin{abstract}
Das Higgs-Teilchen geh�rt zum Higgs-Mechanismus, einer schon in den 1960er Jahren vorgeschlagenen Theorie, nach der alle fundamentalen Elementarteilchen (beispielsweise das Elektron) ihre Masse erst durch die Wechselwirkung mit dem allgegenw�rtigen Higgs-Feld erhalten. Als einziges Teilchen des Standardmodells ist das Higgs-Boson experimentell noch nicht vollst�ndig gesichert.
\end{abstract}
}
%\end{otherlanguage}
