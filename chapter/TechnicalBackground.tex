\chapter{Technical Background}	\label{chap:background}
\section{General processing chain}
%audio - pre-processing - feat. extraction - machine learning
% unseen audio - pre-processing - feat. extraction  - run through model - result
\section{Audio features}
%mfcc, sdc, trap; mention filterbank feats...?
\section{Machine learning algorithms}
This section describes the various machine learning algorithms employed throughout this thesis. Gaussian Mixture Models (GMMs), Hidden Markov Models (HMMs), and Support Vector Machines (SVMs) are three traditional approaches that are used as the basis of many new approaches, and were used for several starting experiments. i-Vector processing is a relatively new, more sophisticated approach that bundles several other machine learning techniques.\\
In recent years, Deep Learning has become the standard for machine learning applications \cite{}. This chapter also describes two of those new approaches that were used in this work: Deep Neural Networks (DNNs) and Deep Belief Networks (DBNs).

\subsection{Gaussian Mixture Models}
\subsection{Hidden Markov Models}
\subsection{Support Vector Machines}
\subsection{i-Vector processing}
\subsection{Artificial Neural Networks}
\subsubsection{Deep Neural Networks}
\subsubsection{Deep Belief Networks}
\section{Evaluation}
\subsection{Evaluation of phoneme recognition tasks}
%alignment?
\subsection{Evaluation of language identification tasks}
\subsection{Evaluation of keyword spotting tasks}
\section{Common application systems}
\subsection{Systems for phoneme recognition}
\subsection{Systems for forced alignment}
\subsection{Systems for language identification}
\subsection{Systems for keyword spotting}




